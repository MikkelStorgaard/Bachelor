\section{Type aliases}
To make an implementation of functors\ref{functors} in Futhark, it was necessary
to implement type aliases first.



\subsection{The language}\label{futharks_types}
The initial type system in Futhark supported the following type definitions:
\begin{align*}
  Type           & = & \texttt{Primitive type} \\
                 &\ | & ( Types ) \\
                 &\ | & [ Type ] \\
  \\
  \\
  Types          & = & Type\ ,\ Types \\
                 &\ | & Type \\
  \\
  \\
  Primitive type & = & UnsignedInteger \\
                 &\ | & SignedInteger \\
                 &\ | & FloatType \\
                 &\ | & Boolean \\
\end{align*}

Implementing type aliases expands our type constructions as following:
\begin{align*}
  Type           & : & \texttt{Primitive type} \\
                 &\ | & ( Types ) \\
                 &\ | & [ Type ] \\
                 &\ | & TypeAlias \\
  \\
  \\
  Types          & : & Type\ ,\ Types \\
                 &\ | & Type \\
  \\
  \\
  Primitive type & : & \texttt{UnsignedInteger} \\
                 &\ | & \texttt{SignedInteger} \\
                 &\ | & \texttt{FloatType} \\
                 &\ | & \texttt{Boolean} \\
  \\
  \\
  TypeAlias      & :  & \texttt{type}\ strid\ \texttt{=}\ Type \\
\end{align*}
where $type\_id$ is short for \textit{string id}, a string identifier.

\subsection{Interference rules}
A type alias in a Futhark program is done like this: \texttt{type} \textit{Strids} \texttt{=} \textit{Type} ,\\
where \textit{Type} is as defined in the grammar above.
\\
\label{typealiasinterference}
\begin{tcolorbox}
Given an environment
$\Gamma : \{FE, TE\},\ Types = \{ strid \to Type \},\ Type \to \tau\footnote{We will let $\tau$ designate a primitive type in Futhark}$
we can define the following interference rules for using type aliases:

\begin{prooftree}
  \AxiomC{$\Gamma \vdash typedecl \Rightarrow \Gamma ' $}
  \AxiomC{$\Gamma \vdash Type \to \tau$}
  \AxiomC{$\Gamma \not \vdash type\_id \to Type$}
    \TrinaryInfC{$\Gamma \vdash$ \texttt{type} \textit{type\_id }\texttt{=}
      \textit{Type} $\Rightarrow$ \{$ \emptyset , \{ type\_id \to Type \}\}
      \oplus \Gamma$}
\end{prooftree}
where
\begin{align*}
             & \{ \emptyset , \{ strid \to Type \} \} \oplus \Gamma  \\
 \Rightarrow & Types(\Gamma) \leftarrow \ Types(\Gamma) \cup \{strid \to Type\}
\end{align*}
\\
iff the type alias declaration does not
clash with the current environment. There is no clash, if the declaration obeys
the three rules in the implementation subsection\ref{typealiasimplementation}.
\\
\\
We can assign the same type to several string ids simultaneously, as long as we don't declare the same type alias twice.
\begin{prooftree}
  \AxiomC{$\Gamma \vdash typedecl \Rightarrow \Gamma ' $}
  \AxiomC{Strids = $strid_1 , ... , strid_n$ }
    \BinaryInfC{ $\Gamma \vdash$\texttt{type }\textit{Strids }\texttt{= }
      \textit{Type} $\Rightarrow$ $\{ strid_1 \to Type, .. , strid_n \to Type \} \oplus \Gamma \Rightarrow \Gamma' \}$}
\end{prooftree}
      where
\begin{align*}
             & \{ strid_1 \to Type, \cdots,  strid_n \to Type, \} \oplus \Gamma  \\
 \Rightarrow & Types(\Gamma) \rightarrow \ Types(\Gamma) \cup \{ strid_1 \to Type, \cdots,  strid_n \to Type \}
\end{align*}
      

\end{tcolorbox}
\subsection{Implementation}\label{typealiasimplementation}
A type alias $strid_i \to Type$ declaration is succesful, if three rules are followed:
\begin{enumerate}
  \item The alias $strid_i$ is not already declared\footnote{This is by convention of
      only being able to define a value \textit{once}} in the current local
    environment. I.e. the example below:
    \begin{verbatim}
      type foo = i32
      type foo = f32
    \end{verbatim}
\clearpage
  \item The alias $strid_i$ refers to a type (or a type alias), that is either already
    defined in the current environment (including structure environments), or is
    in the same declaration chunk as $strid_i$. In the example below, \texttt{foo} refers
    to \texttt{bar,} but \texttt{bar} is in the same chunk as \texttt{foo}.\\
    \\
    Therefore, \texttt{foo} can be resolved by resolving \texttt{bar}. The implementation of this
    described in a later subsection\ref{typealiasingcode}.
    \begin{verbatim} 
      type foo = bar
      type bar = {f32, i32}
    \end{verbatim}
    There are no hard limit to the number of type aliases that has to be
    checked, before a type alias is resolved:
    \label{longchain}
      \begin{verbatim}
        type foo = bar
        type bar = {f32, baz, i32}
        type baz = [{bee, bang, boo}]
        type boo ...
  
        ...
  
        type bep = i32
        \end{verbatim}
    Such a chain of type aliases is allowed, as long as the last of the three
    rules is obeyed:
    \\
    \\
  \item The alias being resolved cannot be cyclically defined.\label{cyclicaldefinitionerror}\\
    Imagine that some type \texttt{type my\_type = foo} is in the chain in the
    type aliasing example\ref{longchain} above.\\

    In this case, the compiler is first trying to resolve \texttt{foo} by
    resolving \texttt{bar,} and trying to resolving \texttt{bar} by resolving \texttt{baz}, et cetera. \\
    At some point, the compiler encounters \texttt{my\_type}, and must resolve \texttt{foo} to
    continue - which creates a cycle, because \texttt{foo} is resolved by \texttt{bar}, and so on.

    To keep track of this, the compiler maintains the set of aliases that has
    been visited in the attempt to resolve some type alias. Every time another
    type alias has to be checked, the compiler first checks the set to find out,
    whether this alias is already on the list of aliases that needs to be
    resolved.

    If so, the compiler returns an error.
    The implementation of this can be read here\ref{typealiasimplementation}
\end{enumerate}
\subsection{Parsing a type alias}
\subsubsection{Data types for describing a type}
Initially, the types parsed in a Futhark program were always represented as instances of the
following datatype \texttt{TypeBase}:
\begin{lstlisting}[language=Haskell]

  data TypeBase shape as vn = Prim PrimType
                            | Array (ArrayTypeBase shape as vn)
                            | Tuple [TypeBase shape as vn]

  data ArrayTypeBase shape as vn =
        PrimArray PrimType (shape vn) Uniqueness (as vn)
      -- ^ An array whose elements are primitive types.
      | TupleArray [TupleArrayElemTypeBase shape as vn] (shape vn) Uniqueness
      -- ^ An array whose elements are tuples.

  data TupleArrayElemTypeBase shape as vn =
    PrimArrayElem PrimType (as vn) Uniqueness
  | ArrayArrayElem (ArrayTypeBase shape as vn)
  | TupleArrayElem [TupleArrayElemTypeBase shape as vn]
\end{lstlisting}
\noindent
These are also the data types that are used in the internal Futhark program.
\\
Before this project, Futhark parsed types from Futhark source code directly to
TypeBases.
However, we decided to add an intermediate data type between raw source
code and TypeBases, to make type aliases available.

\begin{lstlisting}[language=Haskell]
data UserType vn = UserPrim PrimType SrcLoc
                 | UserArray (UserType vn) (DimDecl vn) SrcLoc
                 | UserTuple [UserType vn] SrcLoc
                 | UserTypeAlias LongName SrcLoc
                 | UserUnique (UserType vn) SrcLoc

    deriving (Show)
\end{lstlisting}
\noindent
The parser was changed, so type declarations in Futhark source would now be parsed as UserTypes and not
typebases.
\pagebreak
\subsubsection{Adding resolved types to scope}\label{typealiasingcode}
\begin{lstlisting}[language=Haskell, numbers=left]
  type TypeAliasTableM =
  ReaderT (HS.HashSet LongName) (StateT Scope TypeM)

typeAliasTableFromProg :: [TypeDefBase NoInfo VName]
                       -> Scope
                       -> TypeM Scope
typeAliasTableFromProg defs scope = do
  checkForDuplicateTypes defs
  execStateT (runReaderT (mapM_ process defs) mempty) scope
  where
        findDefByName name = find ((==name) . typeAlias) defs

        process :: TypeDefBase NoInfo VName
                -> TypeAliasTableM (StructTypeBase VName)
        process (TypeDef name (TypeDecl ut NoInfo) _) = do
          t <- expandType typeOfName ut
          modify $ (addType name t)
          return t

        typeOfName :: LongName -> SrcLoc
                   -> TypeAliasTableM (StructTypeBase VName)
        typeOfName (prefixes, name) loc = do
          inside <- ask
          known <- get
          case typeFromScope (prefixes, name) known of
            Just t -> return t
            Nothing
              | (prefixes, name) `HS.member` inside ->
                  throwError $ CyclicalTypeDefinition loc name
              | Just def <- findDefByName name ->
                  local (HS.insert (prefixes, name)) $ process def
              | otherwise ->
                  throwError $ UndefinedAlias loc name
\end{lstlisting}
\noindent
TypeAliasTableM is a monad stack that is used to resolve a list of type alias
declarations in a declaration chunk. \\
It is a reader monad transformer that contains a state monad transformer, that
contains the TypeM monad.
\\
\\
\noindent
The reader monad is used to contain a HashSet as its environment. This
environment is used to keep check of cyclical definitions as described in
\ref{cyclicaldefinitionerror}.
\\
\\
For each type alias, we use the function \texttt{process} to resolve the type, and modify
the scope contained within the state monad of the ReaderT.
\\
\\
Resolving a type from a type aliasing is done using the function \texttt{typeOfName} in
the code snippet\ref{typealiasingcode}. \texttt{typeOfName} tries to
resolve the type by name by retrieving the scope contained in the transformed
State monad inside the reader.
\clearpage
\noindent
If this is not immediately possible, we must either continue our search for the type, throw an error due to a cyclical type definition, or throw an error because the type alias has not been defined yet.
\\
If our attempt to resolve $strid_a$ leads to another alias
$strid_b$\footnote{line 27-33}, we add
$strid_a$ to our reader environment using the function \texttt{local}, and
process $alias_b$ instead.
\\
\\
\begin{tcolorbox}
\textbf{Acknowledgement}: The initial design of \texttt{expandType} and the addition of
type aliases to the scope was initially much larger, but was reduced in size by
Troels Henriksen, who rewrote the process to use monads, and reduced some code
duplication. 
\end{tcolorbox}
\subsubsection{Converting UserType to TypeBase}
\begin{lstlisting}[language=Haskell, numbers=left]
expandType :: (Applicative m, MonadError TypeError m) =>
               (LongName -> SrcLoc -> m (StructTypeBase VName))
            -> UserType VName
            -> m (StructTypeBase VName)

expandType look (UserTypeAlias longname loc) =
  look longname loc
expandType _ (UserPrim prim _) =
  return $ Prim prim
expandType look (UserTuple ts _) =
  Tuple <$> mapM (expandType look) ts
expandType look (UserArray t d _) = do
  t' <- expandType look t
  return $ arrayOf t' (ShapeDecl [d]) Nonunique
expandType look (UserUnique t loc) = do
  t' <- expandType look t
  case t' of
    Array{} -> return $ t' `setUniqueness` Unique
    _       -> throwError $ InvalidUniqueness loc $ toStructural t'

\end{lstlisting}

\subsubsection{Why we added UserType instead of extending TypeBase}
Adding UserType and then resolving these into TypeBases whilst running the
program through the TypeChecker, removes the need of handling UserAliases after
the type check, where these aliases are resolved.
\\\\
\label{typeclarification}Furthermore, not all information about a TypeBase declaration
can actually be claimed already at program parse time.
Some information about i.e. IKKE SIKKER HER array dimensionality in regards to aliased arrays, is decided within the type
checker as well. 

\subsubsection{The slip from type aliases to realized types}
Since every type alias is resolved in the type checker, the UserTypes are not
used after the type check.

The data type for a type declaration is this:
\begin{lstlisting}[language=Haskell]
data TypeDeclBase f vn =
  TypeDecl { declaredType :: UserType vn
                             -- ^ The type declared by the user.
           , expandedType :: f (StructTypeBase vn)
                             -- ^ The type deduced by the type checker.
           }
\end{lstlisting}
An unresolved type looks like this:\\\\
\indent \texttt{TypeDecl userType NoInfo}\\
\\
After resolve, NoInfo has been filled out with a variable of type\\
\texttt{Info TypeBase}, giving us the following TypeDecl:\\\\
\indent \texttt{TypeDecl usertype (Info typebase)}.\\
\\
At any point after the type check, only the \texttt{expandedType} of TypeDecl is used.

\subsection{Results}
The addition of type aliases works without any issues.
To verify this, Futhark has been tested to pass all of the tests in futharks
test suite\footnote{located in folder \texttt{futhark/data/tests}}.

However, it was also necessary to write tests to specifically confirm, that the
implementation respects the rules defined in \ref{typealiasimplementation}

\subsubsection{A type cannot be defined twice in the same environment}
From \texttt{alias-error3.fut} in \texttt{futhark/src/data/tests/types}:
\begin{verbatim}
-- You may not define the same alias twice.
--
-- ==
-- error: Duplicate.*mydup

type mydup = int
type mydup = f32

fun int main(int x) = x
\end{verbatim}

This program fails as expected.

\subsubsection{A type alias cannot be defined, if it refers to a type alias that has not been defined}
From \texttt{alias-error4.fut} in \texttt{futhark/src/data/tests/types}:
\begin{verbatim}
-- No undefined types!
--
-- ==
-- error: .*not defined.*

type foo = bar

fun foo main(foo x) = x
\end{verbatim}
This program fails as expected.

\subsubsection{A type alias cannot be cyclically defined}
From \texttt{alias-error5.fut} in \texttt{futhark/src/data/tests/types}:
\begin{verbatim}
-- No tricky circular types!
--
-- ==
-- error: .*cycl.*

type t0 = [t1]
type t1 = {int, float, t2}
type t2 = t0

fun t1 main(t1 x) = x
\end{verbatim}
This program fails as expected.
\clearpage
\subsubsection{Example of planet simulations being simplified by type aliases}\label{nbody}
A nice example of the benefits of type aliasing, is the N-body simulation
(\texttt{nbody}), which is a simulation over the n-body
problem\footnote{\url{https://en.wikipedia.org/wiki/N-body\_problem}}.

The original Futhark implementation of the simulation contained function
arguments of tuples with arity up to 10. Whilst it is still necessary to bring
all the arguments throughout the program, type aliasing makes the program itself
much more readable:
\begin{lstlisting}
 -- N-body simulation based on the one from Accelerate:
-- https://github.com/AccelerateHS/accelerate-examples/tree/master/examples/n-body
--
-- Type descriptions:
--
-- type mass = f32
-- type position = {f32, f32, f32}
-- type acceleration = {f32, f32, f32}
-- type velocity = {f32, f32, f32}
-- type body = (position, mass, velocity, acceleration)
--           =~ {f32, f32, f32, -- position
--               f32,           -- mass
--               f32, f32, f32, -- velocity
--               f32, f32, f32} -- acceleration
--

fun {f32, f32, f32}
  vec_add({f32, f32, f32} v1,
               {f32, f32, f32} v2) =
  let {x1, y1, z1} = v1
  let {x2, y2, z2} = v2
  in {x1 + x2, y1 + y2, z1 + z2}

fun {f32, f32, f32}
  vec_subtract({f32, f32, f32} v1,
               {f32, f32, f32} v2) =
  let {x1, y1, z1} = v1
  let {x2, y2, z2} = v2
  in {x1 - x2, y1 - y2, z1 - z2}

fun {f32, f32, f32}
  vec_mult_factor(f32 factor,
                  {f32, f32, f32} v) =
  let {x, y, z} = v
  in {x * factor, y * factor, z * factor}

fun f32
  dot({f32, f32, f32} v1,
      {f32, f32, f32} v2) =
  let {x1, y1, z1} = v1
  let {x2, y2, z2} = v2
  in x1 * x2 + y1 * y2 + z1 * z2
  
fun {f32, f32, f32}
  accel(f32 epsilon,
        {f32, f32, f32} pi,
        f32 mi,
        {f32, f32, f32} pj,
        f32 mj) =
  let r = vec_subtract(pj, pi)
  let rsqr = dot(r, r) + epsilon * epsilon
  let invr = 1.0f32 / sqrt32(rsqr)
  let invr3 = invr * invr * invr
  let s = mj * invr3
  in vec_mult_factor(s, r)

fun {f32, f32, f32}
  accel_wrap(f32 epsilon,
             {f32, f32, f32, f32, f32, f32, f32, f32, f32, f32} body_i,
             {f32, f32, f32, f32, f32, f32, f32, f32, f32, f32} body_j) =
  let {xi, yi, zi, mi, _, _, _, _, _, _} = body_i
  let {xj, yj, zj, mj, _, _, _, _, _, _} = body_j
  let pi = {xi, yi, zi}
  let pj = {xj, yj, zj}
  in accel(epsilon, pi, mi, pj, mj)
  
fun {f32, f32, f32}
  move(f32 epsilon,
       [{f32, f32, f32, f32, f32, f32, f32, f32, f32, f32}] bodies,
       {f32, f32, f32, f32, f32, f32, f32, f32, f32, f32} body) =
  let accels = map(fn {f32, f32, f32} ({f32, f32, f32, f32, f32, f32, f32, f32, f32, f32} body_other) =>
                     accel_wrap(epsilon, body, body_other),
                   bodies)
  in reduceComm(vec_add, {0f32, 0f32, 0f32}, accels)

fun [{f32, f32, f32}]
  calc_accels(f32 epsilon,
              [{f32, f32, f32, f32, f32, f32, f32, f32, f32, f32}] bodies) =
  map(move(epsilon, bodies), bodies)

fun {f32, f32, f32, f32, f32, f32, f32, f32, f32, f32}
  advance_body(f32 time_step,
               {f32, f32, f32, f32, f32, f32, f32, f32, f32, f32} body) =
  let {xp, yp, zp, mass, xv, yv, zv, xa, ya, za} = body
  let pos = {xp, yp, zp}
  let vel = {xv, yv, zv}
  let acc = {xa, ya, za}
  let pos' = vec_add(pos, vec_mult_factor(time_step, vel))
  let vel' = vec_add(vel, vec_mult_factor(time_step, acc))
  let {xp', yp', zp'} = pos'
  let {xv', yv', zv'} = vel'
  in {xp', yp', zp', mass, xv', yv', zv', xa, ya, za}
  
fun {f32, f32, f32, f32, f32, f32, f32, f32, f32, f32}
  advance_body_wrap(f32 time_step,
                    {f32, f32, f32, f32, f32, f32, f32, f32, f32, f32} body,
                    {f32, f32, f32} accel) =
  let {xp, yp, zp, m, xv, yv, zv, _, _, _} = body
  let accel' = vec_mult_factor(m, accel)
  let {xa', ya', za'} = accel'
  let body' = {xp, yp, zp, m, xv, yv, zv, xa', ya', za'}
  in advance_body(time_step, body')
  
fun [{f32, f32, f32, f32, f32, f32, f32, f32, f32, f32}, n]
  advance_bodies(f32 epsilon,
                 f32 time_step,
                 [{f32, f32, f32, f32, f32, f32, f32, f32, f32, f32}, n] bodies) =
  let accels = calc_accels(epsilon, bodies)
  in zipWith(advance_body_wrap(time_step), bodies, accels)

fun [{f32, f32, f32, f32, f32, f32, f32, f32, f32, f32}, n]
  advance_bodies_steps(i32 n_steps,
                       f32 epsilon,
                       f32 time_step,
                       [{f32, f32, f32, f32, f32, f32, f32, f32, f32, f32}, n] bodies) =
  loop (bodies) = for i < n_steps do
    advance_bodies(epsilon, time_step, bodies)
  in bodies

fun {[f32, n], [f32, n], [f32, n], [f32, n], [f32, n], [f32, n], [f32, n], [f32, n], [f32, n], [f32, n]}
  main(i32 n_steps,
       f32 epsilon,
       f32 time_step,
       [f32, n] xps,
       [f32, n] yps,
       [f32, n] zps,
       [f32, n] ms,
       [f32, n] xvs,
       [f32, n] yvs,
       [f32, n] zvs,
       [f32, n] xas,
       [f32, n] yas,
       [f32, n] zas) =
  let bodies = zip(xps, yps, zps, ms, xvs, yvs, zvs, xas, yas, zas)
  let bodies' = advance_bodies_steps(n_steps, epsilon, time_step, bodies)
  in unzip(bodies')
\end{lstlisting}

to
\clearpage
\noindent \textbf{nbody using type aliases:}
\begin{lstlisting}
type mass = f32
type vec3 = (f32, f32, f32)
type position, acceleration, velocity = vec3

type body = (position, mass, velocity, acceleration)

fun vec3 vec_add(vec3 v1, vec3 v2) =
  let (x1, y1, z1) = v1
  let (x2, y2, z2) = v2
  in (x1 + x2, y1 + y2, z1 + z2)

fun vec3 vec_subtract(vec3 v1, vec3 v2) =
  let (x1, y1, z1) = v1
  let (x2, y2, z2) = v2
  in (x1 - x2, y1 - y2, z1 - z2)

fun vec3 vec_mult_factor(f32 factor, vec3 v) =
  let (x, y, z) = v
  in (x * factor, y * factor, z * factor)

fun f32 dot(vec3 v1, vec3 v2) =
  let (x1, y1, z1) = v1
  let (x2, y2, z2) = v2
  in x1 * x2 + y1 * y2 + z1 * z2

fun velocity accel(f32 epsilon, vec3 pi, f32 mi, vec3 pj, f32 mj) =
  let r = vec_subtract(pj, pi)
  let rsqr = dot(r, r) + epsilon * epsilon
  let invr = 1.0f32 / sqrt32(rsqr)
  let invr3 = invr * invr * invr
  let s = mj * invr3
  in vec_mult_factor(s, r)

fun vec3 accel_wrap(f32 epsilon, body body_i, body body_j) =
  let (pi, mi, _ , _) = body_i
  let (pj, mj, _ , _) = body_j
  in accel(epsilon, pi, mi, pj, mj)

fun position move(f32 epsilon, [body] bodies, body this_body) =
  let accels = map(fn acceleration (body other_body) =>
                     accel_wrap(epsilon, this_body, other_body),
                   bodies)
  in reduceComm(vec_add, (0f32, 0f32, 0f32), accels)

fun [acceleration] calc_accels(f32 epsilon, [body] bodies) =
  map(move(epsilon, bodies), bodies)

fun body advance_body(f32 time_step, body this_body) =
  let (pos, mass, vel, acc) = this_body
  let pos' = vec_add(pos, vec_mult_factor(time_step, vel))
  let vel' = vec_add(vel, vec_mult_factor(time_step, acc))
  let (xp', yp', zp') = pos'
  let (xv', yv', zv') = vel'
  in (pos', mass, vel', acc)

fun body advance_body_wrap(f32 time_step, body this_body, acceleration accel) =
  let (pos, mass, vel, acc) = this_body
  let accel' = vec_mult_factor(mass, accel)
  let body' = (pos, mass, vel, accel')
  in advance_body(time_step, body')

fun [body, n] advance_bodies(f32 epsilon, f32 time_step, [body, n] bodies) =
  let accels = calc_accels(epsilon, bodies)
  in zipWith(advance_body_wrap(time_step), bodies, accels)

fun [body, n] advance_bodies_steps(i32 n_steps, f32 epsilon, f32 time_step,
                                   [body, n] bodies) =
  loop (bodies) = for i < n_steps do
    advance_bodies(epsilon, time_step, bodies)
  in bodies

fun body wrap_body (f32 posx, f32 posy, f32 posz,
                    f32 mass,
                    f32 velx, f32 vely, f32 velz,
                    f32 accx, f32 accy, f32 accz) =
  ((posx, posy, posz), mass, (velx, vely, velz), (accx, accy, accz))

fun (f32, f32, f32, f32, f32, f32, f32, f32, f32, f32) unwrap_body(body this_body) =
  let ((posx, posy, posz), mass, (velx, vely, velz), (accx, accy, accz)) = this_body
  in (posx, posy, posz, mass, velx, vely, velz, accx, accy, accz)



fun ([f32, n], [f32, n], [f32, n], [f32, n], [f32, n], [f32, n], [f32, n], [f32, n], [f32, n], [f32, n])
  main(i32 n_steps,
       f32 epsilon,
       f32 time_step,
       [f32, n] xps,
       [f32, n] yps,
       [f32, n] zps,
       [f32, n] ms,
       [f32, n] xvs,
       [f32, n] yvs,
       [f32, n] zvs,
       [f32, n] xas,
       [f32, n] yas,
       [f32, n] zas) =
  let bodies  = map(wrap_body, zip(xps, yps, zps, ms, xvs, yvs, zvs, xas, yas, zas))
  let bodies' = advance_bodies_steps(n_steps, epsilon, time_step, bodies)
  let bodies'' = map(unwrap_body, bodies')
   in unzip(bodies'')
\end{lstlisting}
\subsection{Future work}
To make functors work, it must be possible to declare an abstract type alias in
a structure.\\
\\
That will allow for a structure definition, where a type variable\footnote{in the form of a \textit{strid}} has been declared, but not defined, until the containing structure is instantiated using a functor.
\\
This has not been implemented yet.
\clearpage