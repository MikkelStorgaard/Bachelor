% pøls
\section{Functors}
\label{sec:functors}
A function takes a number of arguments of specified types, and returns a value of some return type.
A \textbf{functor} takes a structure\ref{structures} of a specified signature,
and returns a structure of some signature.

The concept is best explained through an example:
\begin{lstlisting}[language=Haskell]
  --
  --==
  -- input {}
  -- output { 4 }

  sig NumType {
    type t
    val add : (t,t) -> t
    val subtract : (t,t) -> t
    val divide : (t,t) -> t
    val multiply : (t,t) -> t
  }

  struct Int {
    type t = int
    fun t add(t a, t b) = a + b
    fun t subtract(t a, t b) = a - b
    fun t divide(t a, t b) = a / b
    fun t multiply(t a, t b) = a * b
    
  } : NumType

  functor Doubler(number : NumType){
    type t = number.t
    fun t add(t a, t b) = number.add
    fun t double(t a) = add(a,a) 
  }

  struct IntDoubler = Doubler(Int)

  fun int main() = IntDoubler.double(2)
\end{lstlisting}
We first declare a signature, which defines the number type. We declare it so
that there is a type t, that has four standard operations 


%% how it works



%% the reason is that WE CAN HAVE ALMOST POLYMORPHISM


