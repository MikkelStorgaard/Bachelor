To work with type aliases and modules, we need to define the internal
environment of Futhark during compile time.

Before starting this project, the environment of Futhark could be described like
this:
\equation{$Env = Functions$}, where any $\forall Function \in Functions,
Function : (name, ArgTypes, RetType)$ where both ArgTypes and RetTypes they
have types $\tau$, which is any type as defined in
\ref{initialtypegrammar}.\\

It is the goal of this project to expand the environment of Futhark, so that
\equation{$Env = (Functions, TypeAliases, Signatures, Structures)$}.

TypeAliases, Signatures and Structures are described in their respective
sections. A structure can be regarded as a structure name, a reference to the
current total environment and an environment contained in the structure. that $Structure name = (name, Env,  Env_{name})$

\subsection{Interference rules}
Let us define some environment behaviour of Futhark at compile time.
If a function definition is defined as this: $Function : fun RetType
name(ArgTypes) = Expressions$ , then adding a function to the Futhark environment follows the following rule:
\begin{prooftree}
  \AxiomC{Env}
  \AxiomC{name \not\in Functions_{Env}
    \BinaryInfC{Env $\vdash$ $Function_{name}$ \implies Env' where Env' = Env \cup Function_{name}}
\end{prooftree}

After expanding Env to contain the three new elements, we can define the
addition of any of these three elements follow the following rule:
\begin{prooftree}
  \AxiomC{Env}
  \AxiomC{name \not\in Signatures_{Env} \u
    \BinaryInfC{Env $\vdash$ $Signature_{name}$ \implies Env' where Env' = Env \cup Signature_{name}}
\end{prooftree}

\begin{prooftree}
  \AxiomC{Env}
  \AxiomC{name \not\in Structures_{Env}
    \BinaryInfC{Env $\vdash$ $Structure_{name}$ \implies Env' where Env' = Env \cup Structure_{name}}
\end{prooftree}

\begin{prooftree}
  \AxiomC{Env}
  \AxiomC{name \not\in TypeAliases_{Env}
    \BinaryInfC{Env $\vdash$ $TypeAlias_{name}$ \implies Env' where Env' = Env \cup TypeAlias_{name}}
\end{prooftree}


\subsection{Accessing types and functions within structures}
The modules implemented in this project uses variable shadowing.
When a new structure is defined, \textit{variable shadowing} grants this
structure access to all previously defined functions, types and modules up until
the definition of this structure.\\

Accessing types and functions are necessary in two cases: when retrieving a type
definition from a structure, and when using an
\textit{Apply}-expression\footnote{An Apply-expression is the application of a
  function, i.e \texttt{foo(a,b)} in \texttt{let result = foo(a,b) }} in a
Futhark program.


\subsubsection{Chunk-wise parsing of the Futhark program}
The declarations in a Futhark program are type checked and added to the internal program in
chunks.
These chunks are divided by structure declarations.

\begin{figure}
  \begin{verbatim}
    type foo = M0.bar
    structure M0 =
      struct
        type bar = int
      end
  \end{verbatim}
  \caption{This is not allowed, because the compiler does not know about M0,
    when foo is type checked}
\end{figure}

\begin{figure}
  \begin{verbatim}
    structure M0 =
      struct
        type bar = int
      end
    type foo = M0.bar
  \end{verbatim}
  \caption{This is allowed, because the compiler has already checked M0,
    when foo is being type checked}
\end{figure}

A structure is not added to the current environment, before it has been parsed
completely:
\begin{figure}
  \begin{verbatim}
  structure M0 =
    struct
      type foo = int
      type bar = M0.foo
    end
  \end{verbatim}
\end{figure}
The code above will return an \texttt{UndefinedStructure} error, as M0 has not
been added to the environment when bar is attempted to be resolved}

If the compiler encounters a reference to a value that has not yet been defined
in the local environment, it will attempt to resolve the reference in the
toplevel environment.

Look at the following example:

\begin{figure}
\begin{verbatim}
structure M0 =
  struct
    type foo = f32
  end

structure M1 =
  type bar = M0.foo
  structure M0 =
    struct
      type foo = int
    end
  type baz = M0.foo
  end
\end{verbatim}
  \caption{The identifier M0 can refer to two different structures}
\end{figure}
Type \texttt{bar} has the type \texttt{f32}, whilst \texttt{baz} has the type
\texttt{int}.
This is because \texttt{bar} does not know about \texttt{M1s} nested structure
\texttt{M0}, as opposed to \texttt{baz}, which does.

\begin{verbatim}
type foo = {int, f32}

structure M0 =
  struct
    type foo = foo -- the type is defined from l. 1
    type bar = f32
  end

structure M1 =
  struct
    type foo = f32
    type bar = M0.bar -- type is defined from l.6

    structure M0 =
      struct
        type foo = M0.foo -- is defined at l. 5
        type bar = {int, int, int}
      end

    type baz = M0.bar -- defined at line 17
  end

type baz = M1.bar -- is defined at l. 13
\end{verbatim}

\subsection{Module access and variable shadowing}

Comments:
\begin{itemize}
  \item 2-4: It is allowed to use the same idenfitier several times in a
    program, as long as they are not used for the same construction; i.e.
    \begin{verbatim}
      type foo = ...
      fun f32 foo = ...
    \end{verbatim} is allowed, as functions and types are not read to the same
    table internally during compilation.
  \end{itemize}

