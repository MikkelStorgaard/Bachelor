\subsection{Abstract}
\label{subsec:abstract}
mere bedre

\section{Introduction}
\label{sec:introduction}
mere bedre

\subsection{Problem definition}
\label{subsec:probdef}
Is it possible to implement a module system in Futhark, which displays features
comparable to the module system implemented in Standard ML? \cite{sml modules}
%% sml-henvisninger fra bogen måske?

\section{Motivation}
\label{subsec:label}
The implementation of a module system, will greatly expand the usability of
Futhark, without having side effects on the performance of the compiled Futhark code.
The following subsections contain the features this project is implementing, and
examples of why these features are desirable.\\

To be able to claim, that modules increases usability, we must first
setup some metrics of usability.

\textbf{Readability increases usability:}\\
%% skriv mere her
\textbf{Compartmentalization of functionality increases usability:}\\
%% skriv mere her
\textbf{A module system increases the ability to share code between developers.}
%% skriv mere her

\subsection{Abstraction increases readability.} In example, type aliasing lets us abstract
from the actual definition of data types.
When writing our source code, we can define our type aliases before writing the
rest of our program.
%% skriv bedre
If we want to define a sphere in a three dimensional space, we want to define it
with 1) a radius, 2) a position , and 3) a direction that it is
moving. 

Let us define a function that multiplies the speed of the sphere by a factor k.
\begin{verbatim}
fun {f32, {f32, f32, f32}, {f32, f32, f32}} multiply_velocity({f32 r, {f32 x_pos , f32 y_pos , f32 z_pos} , {f32 x_dir,
f32 y_dir, f32 z_dir}} , f32 k) =
  {r, {x_pos, y_pos, z_pos}, {k * x_dir , k * y_dir, k * z_dir }}
\end{verbatim}

With type aliasing, we can compartmentalize the data type, and remove the need
for explicitally typing out every parameter of function input.
Coupled with helper functions, we can now multiply the speed of the sphere like
this:

\begin{verbatim}
type vec3 = {f32, f32, f32}
type position, direction = vec3 
type radius = f32
type sphere = {radius, position, direction}

fun vec3 multiply_vector(vec3 {pos_x, pos_y, pos_z}, f32 k) =
  {k * pos_x , k * pos_y , k * pos_z}

fun sphere multiply_velocity(sphere {radius, position, direction}, f32 factor) =
  let new_direction = multiply_vector(direction, factor)
  in {radius, position, new_direction}
\end{verbatim}

As we are now using the sphere type as the function argument, we can pattern
match on the type aliased values contained in the sphere type.
Most importantly, the vectors of the sphere are abstracized into a single variables
instead of tuples.

The end result is a shorter, more readable program. %% se bilag sphere test



\subsection{Approximating higher order functionality whilst keeping performance}
\label{subsec:higherorderfunctionality}
It is possible to express higher-order functionality in Futhark, without taking
a performance hit in the compiled Futhark code.
%% reference to slow higher order functionality
We will reiterate on the concept of modules\ref{modulesaregood}, by introducing
the concept of functors.
First we repeat the three-dimensional vector module
from earlier, but without declaring any particular
primitive\ref{typedefinitions} type as the contained type of the vector:
\begin{verbatim}
  structure Vec3 = 
    struct
      type vector = {t, t, t}
      fun vector add( ... ) = ...
      fun vector subtract( ... ) = ...
      fun vector multiply ( ... ) = ...
      fun vector divide ( ... ) = ...
    end
\end{verbatim}

The structure above cannot be used on its own. Type \texttt{t} is not
instantiated, and the module cannot be type checked, which causes an error.

We can solve that problem, by instantiating the abstract structure, using a
simple functor; the \texttt{where}-clause.
\begin{verbatim}
  structure IntVec3 = Vec3 where type t = int
\end{verbatim}
We can now access the structure \texttt{IntVec3} throughout the rest of the
program. The structure \texttt{IntVec3} is \texttt{Vec3}, except all instances
of type \texttt{t} in \texttt{Vec3} is exchanged with type \texttt{int}.

To recap: functors allows us to define an abstract implementation of some
structure \textbf{ONCE}, and lets us instantiate this structure any number of
times, each time with our own type.

This functionality will be elaborated on in a later section\ref{functors}. 

From a performance-concerned point of view, the module system is desirable.
Every function in the written program, whether it is inside a structure
or defined in the top level program, is ultimately accumulated into the same
scope.

No matter how complex the written Futhark program is, module use and all, the
internalised\footnote[]{The internal representation of the Futhark program
  within the compiler} Futhark program keeps every function in the same table.

The entire program is then compiled, and the optimizations that makes FUTHARK \texttt{GO
FAST } are applied to the included modules as well as the top level declarations.

\subsection{Scope of project}
\label{subsec:project_scope}
The scope of the project is
\begin{itemize}
  \item to implement a type aliasing system in Futhark
  \item to implement a module system in Futhark, which has:
    \begin{itemize}
      \item The definition of structures containing types and functions
      \item Nested modules; meaning that any structure can contain a structure
      \item A well defined way of referring to types, structures and functions
        contained in a structure
    \end{itemize}
  \item to implement functor functionality, so that Futhark supports the
    definition of abstract structures and concretizations of these structures.
\end{itemize}

\subsection{Success definitions}
\label{subsec:label}

