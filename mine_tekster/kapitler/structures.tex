\section{Structures}
\label{sec:structures}
\textbf{Introduction to structures:}
\label{subsec:structuresaregood}
Without structures, every function and type in futhark shares the same scope.
Implementing modules lets us create functions that are alike, but keeps
distinctions between them.
Take this example of a program with two different vector types:

\begin{verbatim}
type vec3 = {f32, f32, f32}
type vec4 = {f32, f32, f32, f32}

fun vec3 vec3_plus(
  vec3 {a_1, ... , a_3}, 
  vec3 {b_1, ..., b_3}
  ) = {a_1 + b_1, ... , a_3 + b_3}

fun vec4 vec4_plus(
  vec4 {a_1, ... , a_4}, 
  vec4 {b_1, ..., b_4}
  ) = {a_1 + b_1, ... , a_4 + b_4}
\end{verbatim}
\noindent
Let us try compartmentalizing a vector type and its functions into a structure.\\
  In the following example, we have defined two different modules, each
containing a structure,
and a futhark program which includes and utilizes these modules:
\begin{verbatim}
Vec3Float.fut:
  structure Vec3Float = 
    struct
      type vector = {f32, f32, f32}
      fun vector plus( ... ) = ...
      fun vector minus( ... ) = ...
      fun vector multiply ( ... ) = ...
    end

Vec4Float.fut:
  structure Vec4Float = 
    struct
      type vector = {f32, f32, f32, f32}
      fun vector plus( ... ) = ...
      fun vector minus( ... ) = ...
      fun vector multiply ( ... ) = ...
    end
\end{verbatim}
\clearpage
\begin{verbatim}
myprogram.fut:

  include Vec3Float
  include Vec4Float

  type vec3 = Vec3Float.vector
  type vec4 = Vec4Float.vector
  
  fun vec4 foo(vec3 vector) = 
    let {a, b, c} = Vec3.plus(vector, vector)
    in Vec4.multiply({a, b, c, 1.0f} , 4.0f)
\end{verbatim}
\noindent
Whilst it \textit{is} possible to create libraries without a module
implementation \footnote{By including libraries that adds functions to the top
  level environment}, the user runs a risk of running into errors like
\texttt{MulipleDefinitionError}\footnote{Multiple functions of same name
  defined}, if any of the library functions uses any of the names, that the
local user is using as well.
\\\\
The module system removes this hazard, as application of functions and types are
done using \textbf{longnames}, which adds prefixes to names. This way, functions
can have the same name, as long as they do not share the same prefix.

\begin{tcolorbox}
\textbf{Longnames:}\\
A longname consists of any amount of prefixes followed by a dot, followed by the
string id of the desired function or type:
\begin{align*}
LongName & :  & \texttt{prefix.}LongName \\
         &\ | & \texttt{identifier}
\end{align*}

In Futhark we will be using string ids for declarations, and longnames for
the accessing types and functions in structures.


\end{tcolorbox}
\subsection{Accessing types and functions within structures}
To work with type aliases and modules, we need to define the internal
environment of Futhark during compile time.\\
\\
Before starting this project, the environment of Futhark could be described like
this:\\
$\Gamma = (FE)$ , where \textit{FE} is a function environment, mapping function ids to functions:

 $\{funid \to funexpr\}$.
\\\\
It is the goal of this project to expand the environment of Futhark, so that\\[0.2em]
$\Gamma = (FE, TE, SIGE, STRUCTE, FUNCTE)$, where the additions are a type alias environment, a signature environment, a structure environment and a functor environment.\\[0.2em]
Type aliases, signatures, structures and functors are described in their respective
sections.\\
A structure can be regarded as a structure name and an environment contained in the structure, so that
$\{strid \to \Gamma_{strid}\}$ is a mapping in \textit{STRUCTE}.
\clearpage
Any $\Gamma_{strid}$ contains its own function-, typealias-, signature-, structure- and functor declarations.\\
\\

\subsection{Interference rules}\label{structuresinterferencerules}
Now that we can discern between the global environment, and any number of
environments in structures, we must redefine the environment behaviour of Futhark at compile time.
\\
\begin{figure}\label{Rule1}
  \begin{tcolorbox}
    \begin{prooftree}
      \AxiomC{$\Gamma \vdash fundecl \Rightarrow \Gamma'$}
      \AxiomC{$\Gamma \not\vdash fun\_id \to funexpr$}
      \BinaryInfC{
      $\Gamma \vdash \texttt{fun }fun\_id \texttt{ = }funexpr
        \Rightarrow \{
        \{fun\_id \to funexpr\} ,
        \emptyset ,
        \emptyset ,
        \emptyset ,
        \emptyset
        \} \oplus \Gamma$}\\
    \end{prooftree}
    iff the function the function body is otherwise well-formed\footnote{I will not go further into
      the functionality of functions, as it is not within scope of this project.}.
  \end{tcolorbox}
  \caption{Rule for adding a function to the environment}
\end{figure}

After expanding the environment to contain the four new elements, we can define the
addition of any of these four elements follow a generalized rule:
\begin{figure}\label{Rule2generalized}
  \begin{tcolorbox}
    \begin{prooftree}
      \AxiomC{ $\Gamma \vdash Decl_1 \Rightarrow \Gamma' $ }
      \AxiomC{ $\Gamma \not\vdash id \to decltype\ Decl_2$
        \BinaryInfC{ $\Gamma \vdash$ \texttt{decltype} \textit{id }\texttt{=} \textit{decl} $\Rightarrow \{ id \to Decl \} $ }
      \end{prooftree}
    \end{tcolorbox}
\end{figure}
Please note, that several declarations can actually share name \textit{id}, as
long as they don't share declaration type.\\
\\
\subsubsection{Interference rules for adding multiple declarations}
Finally, we want to create a rule for which environment we get, when we state
multiple declarations in a row.
\begin{figure}\label{Rule2multiple}
  \begin{tcolorbox}
    \begin{prooftree}
      \AxiomC{ $\Gamma \vdash Decl_1 \Rightarrow \Gamma_1 $ }
      \AxiomC{ $\Gamma \oplus \Gamma_1 \vdash Decl_2 \Rightarrow \Gamma_2 $}
        \BinaryInfC{ $\Gamma \vdash Decl_1 , Decl_2 \Rightarrow \Gamma_1 \oplus \Gamma_2$}
      \end{prooftree}
      \\
      \\
      For any $\Gamma$, $\Gamma_a \oplus \Gamma_b \Rightarrow \Gamma_a \cup \Gamma_b$
    \end{tcolorbox}
\end{figure}
This rule is covers the run of an entire program, since we can set the initial
declaration in the program as $Decl_1$, use rule \ref{Rule2multiple} to handle
the first declaration in the pair $Decl_1$ and $Decl_2$, and afterwards using
the same rule again on $Decl_2$, $Decl_3$.
\\
This procedure is repeated until the rule reaches the two last statements,
$Decl_{n-1}$, $Decl_n$. 

\subsection{Interference rules for structure definitions}
We must take a closer look at what happens, when a structure is defined.
The definition of a structure in Futhark has the following behaviour.
\subsubsection{Rule for multiple declarations of same name in same local environment:}
Building a structure from a list of \textit{structdecls} allows for several
different declarations in the same environment of the same name, as long as
it is different types of declarations:\\
\\


\subsubsection{Rule for adding a structure to the local environment}
\begin{figure}\label{Rule3}
  \begin{tcolorbox}
    \begin{prooftree}
      \AxiomC{ $\Gamma \vdash strdecl \Rightarrow \Gamma'$ }
      \AxiomC{ $\Gamma \vdash strdecls \Rightarrow \Gamma_{strdecls}$}
      \AxiomC{ $\Gamma \oplus \Gamma_{strdecls} \vdash \Gamma_{strid}$}
        \BinaryInfC{$\Gamma \vdash$ \texttt{struct} \textit{strid}
          \texttt{\{} \textit{strdecls} \texttt{\}}
          $\Rightarrow \{ strid \to \Gamma_{strid}\}$}
    \end{prooftree}.
    where $\Gamma \oplus \Gamma_{strdecls} \Rightarrow \Gamma \cup
    \Gamma_{strdecls}.$
    If there is a shared identifier in both $\Gamma$ and $\Gamma_{strdecls}$,
    the definition in $\Gamma_{strdecls}$ is the definition in $\Gamma_{strid}$.
  \end{tcolorbox}
\end{figure}
This means that Futhark exhibits variable shadowing, where definitions in some
structure are valid in its nested structures, unless they are redefined.

\subsection{Interference rules for interpreting functions and types in an
  environment with structures}\label{interpretingfunctionsandtypeswithstructures}
There are three cases where it is necessary to resolve a function or a type from a
longname:\\
1) When applying a function as an expression in a function expression; i.e.
\begin{verbatim}
  let myNumber = Constants.numberFour()
\end{verbatim}
2) When using a function as an argument in a currying function; i.e.
\begin{verbatim}
  let numbers = [1, 2, 3, 4] in
  let sum = reduce(MathLib.plus , 0 , numbers)
\end{verbatim}
3) When using a type definition from a structure; i.e.
\begin{verbatim}
  type int_pair = Pairs.Int.t 
\end{verbatim}
We can define the interference rule for using a longname in the three different
cases. As all three cases handles resolving a longname the same way, the rule
will only be called for the first case:
\begin{figure}\label{Rule5}
  \begin{prooftree}
    \AxiomC{$\Gamma \vdash \mathit{longname} \to fun\_definition$}
    \UnaryInfC{$\Gamma \vdash let x = longname() \Rightarrow x := \downarrow longname$}
  \end{prooftree}
  where
  \begin{align*}
    \downarrow longname & = & return getFromEnv(longname , \Gamma) \\
  \end{align*}.
\end{figure}
\subsection{Implementation}
A program in the Futhark type checker is defined as a list of unchecked
declarations.
I will not describe the process in details, but will instead
explain specific parts of the code.
I will specifically focus on parts of the code, that corresponds to the
interference rules described earlier.
\\
\subsubsection{The Scope datatype}
To describe the environment in the Futhark compiler, we use the Scope datatype:
\begin{verbatim}
data Scope  = Scope { envVtable :: HM.HashMap VName Binding
                    , envFtable :: HM.HashMap Name FunBinding
                    , envTAtable :: HM.HashMap Name TypeBinding
                    , envModTable :: HM.HashMap Name Scope
                    , envBreadcrumb :: LongName
                    }
\end{verbatim}
Given $\Gamma = (Functions, TypeAliases, Structures)$, then \textit{Functions} is
implemented in \texttt{envFtable}, \textit{TypeAliases} in \texttt{envTAtable} and \textit{Structures} in \texttt{envModTable}.
\subsubsection{checkProg}
The type check is initiated in the functions checkProg and checkProg'.
checkProg' checks the initial top level of declaratations for duplicates.
\begin{lstlisting}[language=Haskell]
  checkProg :: UncheckedProg -> Either TypeError (Prog, VNameSource)
  checkProg prog = do
    checkedProg <- runTypeM initialScope src $ Prog <$> checkProg' (progDecs prog')
    return $ flattenProgFunctions checkedProg
    where
      (prog', src) = tagProg' blankNameSource prog
\\  
  checkProg' :: [DecBase NoInfo VName] -> TypeM [DecBase Info VName]
  checkProg' decs = do
    _ <- checkForDuplicateDecs decs
    (_, decs') <- checkDecs decs
    return decs'
\end{lstlisting}
%$
\subsubsection{Checking for duplicates}
\label{subsec:checkingforduplicates}
As described in rule \ref{Rule4}, we allow multiple declarations of same name,
as long as they don't share type:
In example, checking for two function definitions of the same type is done in
the first case of checkForDuplicateDecs:
\begin{lstlisting}[language=Haskell]
checkForDuplicateDecs :: [DecBase NoInfo VName] -> TypeM ()
checkForDuplicateDecs decs = do
  _ <- foldM_ f HM.empty decs
  return ()
  where
    f known dec@(FunOrTypeDec (FunDec (FunDef _ (name,_) _ _ _ loc))) =
      case HM.lookup name known of
        Just dec'@(FunOrTypeDec (FunDec FunDef{})) ->
          throwError $ DupDefinitionError name loc $ decLoc dec'
        _ -> return $ HM.insert name dec known
\end{lstlisting}
      %$
\subsection{Dividing a Futhark program into chunks}
To facilitate variable shadowing, it was necessary to split a parsed Futhark
program into chunks.
\\
A structure declaration alters the enviroment drastically, by enabling the
following program declarations following the structure to access the environment
of the structure.
Therefore we divide any list of declaration into type- and function
declarations, and structure declarations.
An example is given in figure\ref{chunks}
\begin{figure}\label{chunks}
  \begin{tcolorbox}
    \begin{lstlisting}[language=Haskell]
      fun int four() = 4
\\
      structure M0 =
        struct
          fun int double(int a) = a + a
        end
\\
      fun int main() =
        let x = four() in
        M0.double(x, x)
  \end{lstlisting}
  \texttt{four()} does not have \texttt{M0} available in its local environment, but
    \texttt{main()} does, because \texttt{M0} HAS already been parsed, at the point where
    \texttt{main()} is declared
  \end{tcolorbox}
\end{figure}
\\
This chunking is done recursively on a list of declarations:
\begin{lstlisting}[language=Haskell]
chompDecs :: [DecBase NoInfo VName]
          -> ([FunOrTypeDecBase NoInfo VName], [DecBase NoInfo VName])
chompDecs decs = f ([], decs)
  where f (foo , FunOrTypeDec dec : xs ) = f (dec:foo , xs)
        f (foo , bar) = (foo, bar)
\end{lstlisting}
\subsection{Checking function- and type declarations}
\label{subsec:checkfunortype}
\begin{lstlisting}[language=Haskell, numbers=left]
checkDecs :: [DecBase NoInfo VName] -> TypeM (Scope, [DecBase Info VName])
checkDecs decs = do
\\
    let (funOrTypeDecs, rest) = chompDecs decs
    scopeFromFunOrTypeDecs <- buildScopeFromDecs funOrTypeDecs
    local (const scopeFromFunOrTypeDecs) $ do
      checkedeDecs <- checkFunOrTypeDec funOrTypeDecs
      (scope, rest') <- checkDecs rest
      return (scope , checkedeDecs ++ rest')
\\
checkDecs [] = do
  scope <- ask
  return (scope, [])
\end{lstlisting}
% $
1) The current chunk of declarations is built using chompDecs. \\
2) A scope is built from this chunk, by using the three interference rules in
section \ref{normaldecls}.\\
3) The scope from step 2 is now the local scope in the following function
execution.\\
4) checkFunOrTypeDec does the actual typechecking of the function declaration. \\
5) The remainder of the declarations chunked in chompDecs is checked using
checkDecs. Note, that the first element of \texttt{rest} must be a structure
declaration, due to the implementation of chompDecs.
\subsection{Checking structure declarations}\label{checkingstructuredeclarations}
Please note that structures are currently called modules inside Futharks compiler.
\begin{lstlisting}[language=Haskell]
checkDecs (ModDec modd:rest) = do
  (modscope, modd') <- checkMod modd
  local (addModule modscope) $
    do
      (scope, rest') <- checkDecs rest
      return (scope, ModDec modd' : rest' )
\end{lstlisting}
%$
When checkDecs encounters a \texttt{ModDec}, \texttt{checkMod} is called to resolve the \texttt{ModDec}.
The ModDec defines an environment of functions and type declarations called
modscope, which we add to the local environment by calling \texttt{local
  (addModule modscope)}\\
This part of the code is the implementation of rule\ref{Rule3}.
\begin{lstlisting}[language=Haskell]
checkMod :: ModDefBase NoInfo VName -> TypeM (Scope , ModDefBase Info VName)
checkMod (ModDef name decs loc) =
  local (`addBreadcrumb` name) $ do
    _ <- checkForDuplicateDecs decs
    (scope, decs') <- checkDecs decs
    return (scope, ModDef name decs' loc)
\end{lstlisting}
%$
checkMod first adds a ``breadcrumb'' of the structure's name (\textit{strid}) to
the local environment. This is where the transformation in rule \ref{Rule3} of
$\Gamma_{strdecls} \Rightarrow \Gamma_{strid}$ is implemented.
\\
After the environment has been given its name, the internal declaration of the
structure is read using \texttt{checkDecs}.
\\
\subsection{Resolving the application of a longname}
Resolving the application for a longname is done through two short recursive
functions.
\\
Figure \ref{resolvelongname} shows the implementation of the longname rule
\ref{Rule5}:
\begin{figure}\label{resolvelongname}
  \begin{lstlisting}[language=Haskell]
    type LongName = ([Name], name)

    typeFromScope :: LongName -> Scope -> Maybe TypeBinding
    typeFromScope (prefixes, name) scope = do
      scope' <- envFromScope prefixes scope
      let taTable = envTAtable scope'
      HM.lookup name taTable
    
    funFromScope :: LongName -> Scope -> Maybe FunBinding
    funFromScope (prefixes, name) scope = do
      scope' <- envFromScope prefixes scope
      let taTable = envFtable scope'
      HM.lookup name taTable
    
    
    envFromScope :: [Name] -> Scope -> Maybe Scope
      envFromScope (x:xs) scope =
        case HM.lookup x (envModTable scope) of
          Just scope' -> envFromScope xs scope'
          Nothing -> Nothing
    envFromScope [] scope = Just scope
\end{lstlisting}
\end{figure}
%$

\subsubsection{Including structures, functions and types from other files}\label{structuresincludes}
The Futhark \texttt{include}-statement lets the user include
other files into the current program.

This is implemented by letting the Futhark compiler combine the source code from
all the included files, with the declarations in the main program. The combined
program is then sent passed on through the rest of the compiler.

As the Futhark program with an arbitrary number of includes is merged into a
single program within the compiler, we can compartmentalize a program into
discrete files.

However, this creates a hazard: there might be declarations in the included
code, which has names that overlap with names in the remaining code, so that a
\texttt{DuplicateDefinitionError} is triggered. the included code has any declarations
that results in a duplicate definition error. Refer to
\ref{structuresfuturework} for a possible future solution.

\subsubsection{Keeping track of function names}
Futharks variable shadowing made it necessary to extend the type of the function
declaration type. \\
Initially, a function had only the declared name of the function as an identifier.
However, it was necessary to extend this name into a pair:

\texttt{type FunName = (declaredName :: Name, expandedName :: LongName)}.

When a function is added to the function table during \texttt{buildFtable}, it
is added to the function table together with it's longname. The function's
longname contains the function's name, and the name of the (potentially) nested
structures it was defined in.

\subsection{Tests}
\label{subsec:structuretests}
To verify that the implementation of structures was correct, a series of test
programs were written to verify, whether the interference rules defined for the
structures, actually held in an executed Futhark program.

Having writting the structure implementation myself, I have been able to
consider my testing options. I have had the choice of writing either \textit{white box}
tests, \textit{black box }tests, or both.
\\
\\
In the context of these futhark structures, white box tests are defined as tests
that are designed by a programmer, who has intricate knowledge of the
code implementation of the program features, he is supposed to test.\\
I.e., this means trying to write programs, which tests whether there can be
ambiguities in the parsing of the program, or other general errors in the code.
\\
\\
Black box tests are written without knowledge of the code behind the program.
In the context of Futhark structures, this limits the test writer to test,
whether the rules which are specified in \ref{structuresinterferencerules} holds
in the actual implementation.

In the following tests, programs which break the rules are expected to return
with an error.

\subsubsection{Testing rule for multiple declarations [...]}
The following tests are implemented to test whether rule \ref{Rule4} holds.
\texttt{duplicate\_def0.fut}:
\lstinputlisting[language=Haskell]{./tests/duplicate_def0.fut}
This test passes.
\\
In the following test, the structure foo contains declarations of name foo also.
This is a white box test to ensure, that the implementation handles type, struct
and fun declarations in seperate tables.
\texttt{duplicate\_def1.fut}:
\lstinputlisting[language=Haskell]{./tests/duplicate_def1.fut}
This test passes.
\\
In the following tests, the programmer has attempted to declare functions and
types in the same environment, twice.

\texttt{duplicate\_error0.fut}:
\lstinputlisting[language=Haskell]{./tests/duplicate_error0.fut}
This test passes.
\\
In the following test, the structure foo contains declarations of name foo also.
\texttt{duplicate\_error1.fut}:
\lstinputlisting[language=Haskell]{./tests/duplicate_error1.fut}
This test passes.

\subsubsection{Testing structures can be called as expected}
The following tests are implemented to test that calling structures works as
defined in rule \ref{Rule5}.

\texttt{calling\_nested\_module.fut}:
\lstinputlisting[language=Haskell]{./tests/calling_nested_module.fut}
This test passes.
\\
Currying functions such as map and reduce works as expected:\\
\texttt{map\_with\_structure0.fut}:
\lstinputlisting[language=Haskell]{./tests/map_with_structure0.fut}
This test passes.
\\
Using structures from an include works as expected:\\
\texttt{Vec3.fut}:
\lstinputlisting[language=Haskell]{./tests/Vec3.fut}
\lstinputlisting[language=Haskell]{./tests/triangles.fut}
This test passes.
\subsubsection{Testing rule for variable shadowing}
The following tests are implemented to test whether the rule \ref{Rule4a} holds:
Simple shadowing for functions holds:\\
\texttt{shadowing0.fut}:
\lstinputlisting[language=Haskell]{./tests/shadowing0.fut}
Simple shadowing for types holds:\\
\texttt{shadowing1.fut}:
\lstinputlisting[language=Haskell]{./tests/shadowing1.fut}
This test shows, that structures are only read into scope, after they are fully parsed.
\texttt{shadowing2.fut}:
\lstinputlisting[language=Haskell]{./tests/shadowing2.fut}
and
\texttt{undefined\_structure\_err0.fut}:
\lstinputlisting[language=Haskell]{./tests/undefined_structure_err0.fut}

\subsection{Results}
\label{subsec:structuresresults}
The implementation of structures works, and I am satisfied with the results. As
shown in the tests, the structures behave as prescribed in the rules\ref{structuresinterferencerules}.

To answer the problem definition, it is definitely possible to add module
functionality to Futhark, in terms of structure support.
\subsection{Future work}
\label{subsec:structuresfuturework}
As mentioned in \ref{structureincludes}, it is possible to trigger a
\texttt{DuplicateDefinitionError} from including files into the main program.

Given more time, I would like to extend the Futhark \texttt{include} statement
to support an \texttt{as} statement, so that the inclusion

\texttt{include some\_module as M0}

will include the declarations from \texttt{some\_module} into the futhark program, but not
into the top level declarations. Instead, the declarations will be loaded into a
structure \texttt{M0}, which can then be accessed throughout the rest of the program as
any other module.