\section{Signatures}
Just like a type signature of a function is the definitions of the argument- and
return types of a function, we can define module signatures similarly:\\\\
A module signature specifies the type signatures for a set of declarations.\\
Let us define a signature called \texttt{hasPlus}:
\begin{lstlisting}
  sig hasPlus {
    type t
    val plus : (t,t) -> t
  }
\end{lstlisting}

Then, after this declaration, we can define some structure, and append this
structure declaration with a signature:
\begin{lstlisting}
  include hasPlus
  
  struct M0 {
    ...
  } : hasPlus
\end{lstlisting}
With this signature, we promise that the structure \texttt{M0} matches the declarations
in signature \texttt{hasPlus}.
In the instance of \texttt{M0}, we now require, that \texttt{M0} has a
declaration, where alias \texttt{t} is assigned.\texttt{M0} must also have a
function, that takes two arguments of type \texttt{t}, and returns a value of
type \texttt{t}.

In the following example, two structures are written. \texttt{M0} adheres to the
specifications in \texttt{hasPlus} , but \texttt{M1} does not, which ultimately results in an
error at compile time:

\begin{lstlisting}
  include hasPlus
  
  struct M0 {
    type t = int
    fun t plus(t a, t b) = a + b
  } : hasPlus
  
  struct M1 {
    type t = (int, int)
    fun int plus(t tuple) =
      let (a, b) = tuple
      in a + b
  } : hasPlus
\end{lstlisting}
\subsection{Purpose of module signatures}
By themselves, the ability to add a signature to a structure does not add any
significicant usefulness to a language. Signatures are useful together with
\textit{functors}(Section ~\ref{sec:functors}), so I will refrain from doing further
explanation outside of functor context.


\subsection{Implementation}
\textit{This section is not complete, as I have not implemented a complete and working
signature system at this point.
}

We need a data type which represents a signature.
I chose to define it as a table from some declaration name to a signature
binding:
\begin{lstlisting}[language=Haskell]
type Signature = HM.HashMap Name SigBinding

data SigBinding = FunSignature (Maybe (TypeDeclBase Info VName)) (TypeDeclBase Info VName)
                  -- ^ A function may take arguments, but might also be a
                  constant.
                  -- It always has a return type.

                | TypeSignature (TypeDeclBase Info VName)
                  -- A type signature must resolve to a type.  
                  
                | ModSignature LongName
                -- A structure of signature sig, can itself be demanded to
                -- contain a structure of some ModSignature
\end{lstlisting}

With the Signature defined, we need to extend the data type of the structure to
accomodate the possibility of adding a signature to a structure:

\begin{lstlisting}[language=Haskell]
data ModDefBase f vn = ModDef { modName :: Name
                              , modDecls :: [DecBase f vn]
                              , modDefLocation  :: SrcLoc
                              , modSignature :: Maybe LongName
                              }
\end{lstlisting}
The parsed structure from the futhark program now has either
\texttt{modSignature Nothing} , or \texttt{modSignature (Just }\textit{signature}\texttt{)}.

We update checkMod from subsection ~\ref{checkingstructuredeclarations} accordingly¸ by
calling an extra function \texttt{verifyMod}:

\begin{lstlisting}[language=Haskell]
checkMod :: ModDefBase NoInfo VName -> TypeM (Scope , ModDefBase Info VName)
checkMod (ModDef name decs loc sig) =
  local (`addBreadcrumb` name) $ do
    checkForDuplicateDecs decs
    (scope, decs') <- checkDecs decs
    verifyMod decs' sig loc
    return (scope, ModDef name decs' loc sig)

verifyMod :: [DecBase Info VName] -> Maybe LongName -> TypeM ()
verifyMod _ Nothing _ = return ()
verifyMod decs (Just sig) loc = do
  sigtable <- asks (sigFromScope sig)
  case sigtable of
    Nothing -> throwError $ UndefinedLongName sig loc
    Just t ->
      foldM_ verifyDec decs $ HM.toList t
\end{lstlisting}
%$

For each entry in the signature table, we call \texttt{verifyDec} on this entry to
verify, that decs contain a declaration that satisfies the signature table.

As I have not implemented \texttt{verifyDec} completely yet, this implementation
is not finished.

\subsection{Results}
As the signature implementation is \textit{incomplete}, I cannot claim that it
is possible to implement module signatures in Futhark, by referring to tests.
\\
However, the current implementation lacks \textbf{only} a correct implementation of
\texttt{verifyDec} before it works, barring some corrections of bugs that might
show up during tests.
\\
The ongoing work on this feature can be followed on the \texttt{signatures}-branch in
the public Futhark GitHub repo.
\\
\\
To answer the question in the problem statement; whether it is possible to implement module signatures in
Futhark: \\
Yes, it is. 