% pøls
\section{Functors}
\label{sec:functors}
A function takes a number of arguments of specified types, and returns a value of some return type.
A \textbf{functor} takes a structure (Section \ref{sec:structures}) of a specified signature,
and returns a structure of some signature.
\\
\\
The concept is best explained through an example:
\begin{lstlisting}[language=Haskell]
  --
  --==
  -- input {}
  -- output { 4 }

  sig NumType {
    type t
    val add : (t,t) -> t
    val subtract : (t,t) -> t
    val divide : (t,t) -> t
    val multiply : (t,t) -> t
  }

  struct Int {
    type t = int
    fun t add(t a, t b) = a + b
    fun t subtract(t a, t b) = a - b
    fun t divide(t a, t b) = a / b
    fun t multiply(t a, t b) = a * b
    
  } : NumType

  functor Doubler(number : NumType){
    type t = number.t
    fun t add(t a, t b) = number.add
    fun t double(t a) = add(a,a) 
  }

  struct IntDoubler = Doubler(Int)

  fun int main() = IntDoubler.double(2)
\end{lstlisting}
We first declare a signature NumType, which defines the number type. We declare it so
that there is a type t, that has four standard operations (add, subtract, divide
and multiply.).
\\
\\
We then declare a structure Int, which adheres to the NumType signature:
The add function in Int uses the built-in functionality of the plus operator in
Futhark, but could in principle be defined as whatever the programmer would
like \emph as long as the type checks correctly.
\\
\\
Then, the functor is defined: The functor takes a structure as an argument, and
can then refer to this structure in the declarations of the functor, using
dot-notation.
The functor contains a list of type-, function- and structure declarations, just
like the structure does.
\\
\\
However, none of these declarations has to be defined inside the functor: any of the declarations can be loaded from the functor's argument instead.
\\
\\
In the example above, the Doubler functor contains a function called
\texttt{double}. But \texttt{double} is not defined inside the functor, but will
instead become whatever the \texttt{double} function is defined as, inside the \texttt{number}
argument of the functor.
At the application of the Doubler functor, creating IntDoubler, we can
guarantee, that our types still hold. This is because the function definitions
in the functor has an abstract type\footnote{type t}, which is not instantiated
until the functor can get the type definition from the structure argument.

To boot, we can even bind a signature to functor if we want to. In that case, we
will have to create a functor, that returns a structure of signature
\textit{sig}, when it is applied.

\subsection{The reason for functors}
\label{subsec:reasonfunctors}
As explained in the introduction of this report\textbf{\ref{subsec:higherorderfunctionality}},
functors allows the programmer to write modules with abstract structures,
defining some functionality.
%% the reason is that WE CAN HAVE ALMOST POLYMORPHISM
Functors does not give us actual polymorphism in our programs, but it
\emph{does} allow us to generalize functionality, when it is not tied to any
specific type traits.

Riccardo Pucella demonstrates in his "Notes on Programming SML/NJ"\cite[pp. 59-60]{functors}, how to implement a data structure \texttt{Queue}, using a functor that takes a structure of signature \texttt{Stack} as an argument. The queue is a widely used data structure, which makes an abstract implementation an excellent alternative to implementing a queue for each data type we need.
\subsection{Tentative implementation of functors}
\label{subsec:implementing_functors}
Implementing a simple functor which instantiates a structure on application,
should not be very difficult.

On a base level, building a structure from a functor and its argument is a
question of parsing the functor's declaration in the right order.\\
At the time of the functor declaration of some functor
\texttt{functor Funct(}\textit{argument} \texttt{:}
\textit{signature}\texttt{)\{ \textit{Functdecls} \}} \, we do not
know anything about the types of the declarations in \textit{Functdecls}.

We cannot type check this functor yet, so we will instead build a functor,
which contains similar declarations to what the structure contains.
However, we allow for abstract declarations of both types, functions and
structures, such as:
\begin{lstlisting}
  functor S(str : sig){
    type a = str.t
    fun a foo(a left, a right) = str.bar(left,right)
    struct M0 = str.mystruct
  }
\end{lstlisting}
We also allow for function declarations with calls to abstract functions:
\begin{lstlisting}
  functor S(str : sig){
    fun int baz = 2 + str.function()
  }
\end{lstlisting}

Thus we will not handle the functor any further, until it is attempted to be
applied to a structure somewhere in the program. At the application of the
functor on some structure \textit{str}, \textit{str} is assumed to be in scope.
\\\\
Our goal now is to convert the applied functor into a structure. This will
happen during checkDecs.
After applying the functor, we will not have any need
for the functor declaration.
Instead, we need a ModDec in the rest of the program, because the Futhark
internalises the program functions on basis of function declarations, which are
read from the actual function declarations \texttt{FunDec}, which are stored
either in the top level decl list, or in the declaration list contained in a
Futhark structure.

In the following tentative implementation of functors in the Futhark
typechecker, we have added a Functor definition to the Futhark syntax.\\
Furthermore, ModDef have been expanded to allow functor application, checkMod
have been expanded to allow functor application, and checkFunctor has been added.

\begin{lstlisting}[language=Haskell]
  data ModDefBase f vn = ModDef { modName :: Name
                                , modDecls :: ModInternal f vn
                                , modDefLocation  :: SrcLoc
                                , modSignature :: Maybe LongName
                                }

  data ModInternal f vn = ModDecls :: [DecBase f vn]
                        | Functor :: ( functorName :: LongName
                                     . argumentName :: LongName
                                     )
  
  data FunctorDefBase = FunctorDef { functorName :: Name
                                   , functorArg :: LongName
                                   , functorArgSig :: LongName
                                   , functorDecls :: [FunctorDeclBase f vn]
                                   , functorDefLocation :: SrcLoc
                                   }

  ...
                                     
  checkMod :: ModDefBase NoInfo VName -> TypeM (Scope , ModDefBase Info VName)
  checkMod (ModDef name (ModDecls decs) loc sig) =
    local (`addBreadcrumb` name) $ do
      checkForDuplicateDecs decs
      (scope, decs') <- checkDecs decs
      verifyMod decs' sig loc
      return (scope, ModDef name decs' loc sig)

  checkMod :: ModDefBase NoInfo VName -> TypeM (Scope , ModDefBase Info VName)
  checkMod (ModDef name (Functor (functor, applicant) loc sig) =
    local (`addBreadcrumb` name) $ do
      applyFunctor functor applicant

  
  applyFunctor :: LongName -> LongName -> TypeM (Scope, ModDefBase Info VName)
  applyFunctor functorname structname =
    functor <- functorFromEnv functorname
    argScope <- scopeFromEnv structname    
    verifyScope argScope $ functorArgSig functor
    (local (addModule argScope)) $ do
      decs <- resolveFunctorDecs $ functorDecls functor
      checkMod (ModDef structname (ModDecls decs) (functorSig functor) (functorDefLocation functor))
\end{lstlisting}
% $

As our results from structures (Subsection \ref{subsec:structuresresults}) showed, that structures in
Futhark behaved as we would expect, I will claim that this tentative solution
works as well.
applyFunctor doesn't do much else than resolving the abstract declarations of
the functor, into a concrete set of declarations, which we DO know how to
handle, using \texttt{checkDecs}.