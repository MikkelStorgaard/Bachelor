\subsection{Abstract}
\label{subsec:abstract}


\section{Introduction}
\label{sec:introduction}


\subsection{Problem definition}
\label{subsec:probdef}
Is it possible to implement a module system in Futhark, which displays features
comparable to the module system implemented in Standard ML? \cite{sml modules}
%% sml-henvisninger fra bogen måske?

\subsection{Motivation}
\label{subsec:label}
The implementation of a module system, will greatly expand the usability of
Futhark, without having side effects on the performance of the compiled Futhark code.
\subsubsection{Futhark usability and portability}
\label{subsec:usability}
To be able to claim, that modules increases usability, we must first
setup some definitions of usability.

\textbf{Abstraction increases readability.} In example, type aliasing lets us abstract
from the actual definition of data types.
When writing our source code, we can define our type aliases before writing the
rest of our program.
%% skriv bedre
If we want to define a sphere in a three dimensional space, we want to define it
with 1) a radius, 2) a position , and 3) a direction that it is
moving. 

Let us define a function that multiplies the speed of the sphere by a factor k.
\begin{verbatim}
fun {f32, {f32, f32, f32}, {f32, f32, f32}} multiply_velocity({f32 r, {f32 x_pos , f32 y_pos , f32 z_pos} , {f32 x_dir,
f32 y_dir, f32 z_dir}} , f32 k) =
  {r, {x_pos, y_pos, z_pos}, {k * x_dir , k * y_dir, k * z_dir }}
\end{verbatim}

With type aliasing, we can compartmentalize the data type, and remove the need
for explicitally typing out every parameter of function input.
Coupled with helper functions, we can now multiply the speed of the sphere like
this:

\begin{verbatim}
type vec3 = {f32, f32, f32}
type position, direction = vec3 
type radius = f32
type sphere = {radius, position, direction}

fun vec3 multiply_vector(vec3 {pos_x, pos_y, pos_z}, f32 k) =
  {k * pos_x , k * pos_y , k * pos_z}

fun sphere multiply_velocity(sphere {radius, position, direction}, f32 factor) =
  let new_direction = multiply_vector(direction, factor)
  in {radius, position, new_direction}
\end{verbatim}

As we are now using the sphere type as the function argument, we can pattern
match on the type aliased values contained in the sphere type.
Most importantly, the vectors of the sphere are abstracized into a single variables
instead of tuples.

The end result is a shorter, more readable program. %% se bilag sphere test


\textbf{Module language allows for better compartmentalizing and code sharing} \\
Without modules, every function and type in futhark shares the same scope.
Implementing modules lets us create functions that are alike, but keeps
distinctions between them.
Take this example of a program with two different vector types:

\begin{verbatim}
type vec3 = {f32, f32, f32}
type vec4 = {f32, f32, f32, f32}

fun vec3 vec3_plus(vec3 {a_1, ... , a_3}, vec3 {b_1, ..., b_3}) =
  {a_1 + b_1, ... , a_3 + b_3}

fun vec4 vec4_plus(vec4 {a_1, ... , a_4}, vec4 {b_1, ..., b_4}) =
  {a_1 + b_1, ... , a_4 + b_4}
\end{verbatim}

Let us try compartmentalizing a vector type and its functions into a module
structure.
In the following example, we have defined two different futhark modules, each
containing a structure,
and a futhark program which includes and utilizes these modules:
\begin{verbatim}
Vec3Float.fut:
  structure Vec3Float = 
    struct
      type vector = {f32, f32, f32}
      fun vector plus( ... ) = ...
      fun vector minus( ... ) = ...
      fun vector multiply ( ... ) = ...
    end

Vec4Float.fut:
  structure Vec4Float = 
    struct
      type vector = {f32, f32, f32, f32}
      fun vector plus( ... ) = ...
      fun vector minus( ... ) = ...
      fun vector multiply ( ... ) = ...
    end
  
myprogram.fut:

  include Vec3Float
  include Vec4Float

  type vec3 = Vec3Float.vector
  type vec4 = Vec4Float.vector
  
  fun vec4 foo(vec3 vector) = 
    let {a, b, c} = Vec3.plus(vector, vector)
    in Vec4.multiply({a, b, c, 1.0f} , 4.0f)
\end{verbatim}

Whilst it \textit{is} possible to create libraries without a module
implementation, the user runs a risk of running into errors like
\texttt{MulipleDefinitionError}\footnote{Multiple functions of same name
  defined}, if any of the library functions uses any of the names, that the
local user is using as well.

The module system removes this hazard, as names from an imported module
\texttt{Foo} are accessed through dot-notation, as shown in line %% FIX %%
in the example above.

An obvious use case is including a library of standard modules with the Futhark
language, as the ones displayed in other languages like Standard
ML\cite{sml_standard_lib} or Python\cite{python3_docs}.

%% mere her



%% this is elaborated on during the scope
The function calls \texttt{let res = best_function(arg)} and
\texttt{let res = MyModule.best_function(arg)} are calls to two difference
functions.



\subsubsection{Achieving higher order functionality whilst keeping up performance}
\label{subsec:label}
%% reference to 

\subsubsection{Inspiration from Standard ML}

\subsection{Scope of project}
\label{subsec:label}

\subsection{Success definitions}
\label{subsec:label}

