\section{Introduction}
\label{sec:introduction}
This report describes the efforts in defining a concrete method of extending the Futhark Programming Language with a module system.
A module system is here defined as a programming feature,
which allows the programmer to package functionality into a discrete code
package.
\\\\
After a module has been defined, it can then be shared and utilized in any other
program, by including it in these programs. Widely known modules include
\texttt{NumPy}\footnote{\url{http://www.numpy.org/}}, the Python package for
data modelling and computing, and \texttt{NLTK}\footnote{\url{http://nltk.org}},
the Natural Language Toolkit (also for Python.)
\\
These suites can be downloaded and used in any Python project.
\\
\\
A well written module can become very popular. In some cases, a programmers
choice of programming language for a given project, can be decided by the number and quality of
available, project relevant modules.
\\
\\
\\
The module system design for Futhark has been inspired by the module system
implemented in Standard ML\cite{crash_course}. Not only does the Standard ML
modules enable structures with predefined implementations.
It also defines a system of \textit{functors\footnote{Not to be confused with neither mathematical
  functors or functors from Haskell}}, which defines abstract implementations of
structures.
\\
\\
These abstract implementations of structures can be then be instantiated by
applying the functor on an argument of a fitting signature.
This makes it
possible to write structures that does not commit to one certain type, but
instead inherits the functionality defined in the given argument.
\\
\\
This spares the developers for writing a lot of almost identical code. Refer to
\ref{first_functor_example} for a simple example.
\pagebreak
\subsection{Motivation}
\label{subsec:label}
The Futhark language a very young language, and is therefore still very much in
developement.
Extending this language to support modules would increase the usability of
Futhark, which could be a factor in expanding the user base and user retention
of Futhark.
\\
\\
The implementation of a module system, will greatly expand the usability of
Futhark, without having side effects on the performance of the compiled Futhark code.
The following subsections contain the features this project is implementing, and
examples of why these features are desirable.
\subsubsection{Abstraction increases readability.} 
Type aliasing lets us abstract
from the actual definition of data types.
When writing our source code, we can define our type aliases before writing the
rest of our program.
%% skriv bedre
If we want to define a sphere in a three dimensional space, we want to define it
with \\
\textbf{1) a radius}, \textbf{2) a position} , and \textbf{3) a direction} that it is
moving.\\
Let us define a function that multiplies the speed of the sphere by a factor k:
\begin{lstlisting}
fun {f32, {f32, f32, f32}, {f32, f32, f32}} multiply_velocity(
	{f32 r, 
	{f32 x_pos , f32 y_pos , f32 z_pos} , 
	{f32 x_dir,f32 y_dir, f32 z_dir}} , f32 k) 
	  =
    {r, {x_pos, y_pos, z_pos}, {k * x_dir , k * y_dir, k * z_dir }}
\end{lstlisting}
\noindent
With type aliasing, we can compartmentalize the data type, and remove the need
for explicitally typing out every parameter of function input.
Coupled with helper functions, we can now multiply the speed of the sphere like
this:
\begin{figure}[h]
\begin{verbatim}
type vec3 = {f32, f32, f32}
type position, direction = vec3 
type radius = f32
type sphere = {radius, position, direction}

fun vec3 multiply_vector(vec3 {pos_x, pos_y, pos_z}, f32 k) =
  {k * pos_x , k * pos_y , k * pos_z}

fun sphere multiply_velocity(
  sphere {radius, position, direction}, f32 factor)  =
    let new_direction = multiply_vector(direction, factor)
    in {radius, position, new_direction}
\end{verbatim}
\end{figure}

\noindent As we are now using the sphere type as the function argument, we can pattern
match on the type aliased values contained in the sphere type.
Most importantly, the vectors of the sphere are abstracized into a single variables
instead of tuples.
\\
The end result is a shorter, more readable program~\ref{nbody}

\subsubsection{Compartmentalization of functionality increases usability:}
Splitting code functionality into multiple files, will allow the programmer to
compile and type check these modules individually. The programmer can edit and
contribute to these files independantly of the programs which includes these
modules.

\subsubsection{Approximating higher order functionality whilst keeping performance}
\label{subsec:higherorderfunctionality}
It is possible to express higher-order functionality in Futhark, without taking
a performance hit in the compiled Futhark code.
%% reference to slow higher order functionality
We will reiterate on the concept of modules\ref{sec:structures}, by introducing
the concept of functors.\\
First we repeat the three-dimensional vector module
from earlier, but without declaring any particular
primitive \ref{futharks_types} type as the contained type of the vector:
\begin{verbatim}
  struct Vec3 {
      type vector = {t, t, t}
      fun vector add( ... ) = ...
      fun vector subtract( ... ) = ...
      fun vector multiply ( ... ) = ...
      fun vector divide ( ... ) = ...
  }
\end{verbatim}
\noindent
The structure above cannot be used on its own. Type \texttt{t} is not
instantiated, and the module cannot be type checked, which causes an error.

\noindent
We can solve that problem, by instantiating the abstract structure, using a
simple functor; the \texttt{where}-clause.
\label{first_functor_example}
\begin{verbatim}
  struct Int {
    type t = int 
  }
  struct IntVec3 = Vec3(Int)
\end{verbatim}
\noindent
We can now access the structure \texttt{IntVec3} throughout the rest of the
program.
\\
The structure \texttt{IntVec3} is \texttt{Vec3}, except all instances
of type \texttt{t} in \texttt{Vec3} is exchanged with type \texttt{int}.
\\
\\
\textbf{To recap:} functors allows us to define an abstract implementation of some
structure \textbf{ONCE}, and lets us instantiate this structure any number of
times, each time with our own type.
\\
\\
From a performance-concerned point of view, the module system is desirable, as the Futhark compiler handles all used and included structures at compile time.
\subsection{Problem definition}
\label{subsec:probdef}
Is it possible to implement a module system in Futhark, which displays features
comparable to the module system implemented in Standard ML? \cite{crash_course}.

\subsubsection{Scope of project}
\label{subsec:project_scope}
The scope of the project is
\begin{itemize}
  \item to define and implement a type aliasing system in Futhark
  \item to define and implement a module system in Futhark, which has:
    \begin{itemize}
      \item The definition of structures containing types and functions
      \item Nested modules; meaning that any structure can contain a structure
      \item A well defined way of referring to types, structures and functions
        contained in a structure
    \end{itemize}
  \item to research the possibility of implementing functor functionality, so that Futhark supports the
    definition of abstract structures and concretizations of these structures.
    \\    A suggested design for the implementation of functors should be part of this project.
\end{itemize}
\clearpage

\subsection{Related work}
Futhark modules are designed with Standard MLs module system in mind. Thus, \textit{The
Definition of Standard ML '97} \ref{definition} has been a source of inspiration
for the behaviour of the Futhark module system. \\
However, I have not read the code implementation of SML modules. My code supervisor and I did not find it necessary to lean on the SML implementation of modules:\\
The Futhark compiler could readily support extensions to the language, and it was not difficult to
figure out a way to implement the module system.
\\
\\
Implementing SML modules in other languages is not a unique idea:\\
Amin\ref{scala_modules}Yaraw has shown a way of implementing SML modules in Scala
language. However, he was able to exploit functionality that was already available in
Scala, to create his module system.
\\
\\
Mainly, he was able to use Scala \texttt{functions} as functors to create new Scala
\texttt{classes} (comparable to structures), just as Scala
\texttt{traits} were used for implementing signatures.
\\
\\
Futhark had neither functions (for structures), structures or signatures at the
beginning of the project, which made these necessary to add to the compiler itself.