\documentclass[12pt]{article}
%	options include 12pt or 11pt or 10pt
%	classes include article, report, book, letter, thesis
\usepackage{graphicx}
\usepackage{import}
\usepackage{caption}
\usepackage{hyperref}
\usepackage{subcaption}
\usepackage{pgffor}
\usepackage{amssymb}
\usepackage{listings}
\usepackage{graphicx}
\usepackage{tcolorbox}
\usepackage{subcaption}
\usepackage{amsmath}
\usepackage{bussproofs}
\usepackage[utf8]{inputenc}
\usepackage{dsfont}
\usepackage{relsize}
\usepackage{fancyhdr}
\usepackage{float}
\usepackage{pgffor}
\usepackage{courier}
\usepackage{mdframed}

\usepackage[margin=1in]{geometry}
\lstset{ %
  basicstyle=\footnotesize\ttfamily ,
  breaklines=true ,
  framextopmargin=50pt,
  frame=bottomline
  }
\tcbset{colback=blue!5!white}
\title{
  Modules in Futhark \\
\textit{Bachelor's thesis}
}
\author{Mikkel Storgaard Knudsen}
\pagestyle{fancy}
\begin{document}

\maketitle

\subsection*{Abstract}
\label{subsec:abstract}
This report investigates the benefits and possibilities of implementing a module system in the programming language Futhark. On the basis of the Standard ML module system, both a type aliasing system and a module structure system is succesfully designed (and defined with interference rules) and implemented in Futhark.
Both type aliases and structures are tested in up against their interference rules, to confirm that they indeed adhere to their rules.\\
Finally, tentative implementations of module signatures- and functors are proposed on the background of the work and experience gained during the two first feature implementations of the project.
\\\\
\clearpage
\tableofcontents
\clearpage
\section{Introduction}
\label{sec:introduction}
This report describes the efforts in defining a concrete method of extending the Futhark Programming Language with a module system.
A module system is here defined as a programming feature,
which allows the programmer to package functionality into a discrete code
package.
\\\\
After a module has been defined, it can then be shared and utilized in any other
program, by including it in these programs. Widely known modules include
\texttt{NumPy}\footnote{\url{http://www.numpy.org/}}, the Python package for
data modelling and computing, and \texttt{NLTK}\footnote{\url{http://nltk.org}},
the Natural Language Toolkit (also for Python.)
\\
These suites can be downloaded and used in any Python project.
\\
\\
A well written module can become very popular. In some cases, a programmers
choice of programming language for a given project, can be decided by the number and quality of
available, project relevant modules.
\\
\\
\\
The module system design for Futhark has been inspired by the module system
implemented in Standard ML\cite{crash_course}. Not only does the Standard ML
modules enable structures with predefined implementations.
It also defines a system of \textit{functors\footnote{Not to be confused with neither mathematical
  functors or functors from Haskell}}, which defines abstract implementations of
structures.
\\
\\
These abstract implementations of structures can be then be instantiated by
applying the functor on an argument of a fitting signature.
This makes it
possible to write structures that does not commit to one certain type, but
instead inherits the functionality defined in the given argument.
\\
\\
This spares the developers for writing a lot of almost identical code. Refer to
\ref{first_functor_example} for a simple example.
\\
\\
\textbf{Reader expectations}:\\
The reader is expected to have experience with statically typed languages,
functional programming, and preferable the Haskell programming language.

\pagebreak
\subsection{Motivation}
\label{subsec:label}
The Futhark language a very young language, and is therefore still very much in
developement.
Extending this language to support modules would increase the usability of
Futhark, which could be a factor in expanding the user base and user retention
of Futhark.
\\
\\
The implementation of a module system, will greatly expand the usability of
Futhark, without having side effects on the performance of the compiled Futhark code.
The following subsections contain the features this project is implementing, and
examples of why these features are desirable.
\subsubsection{Abstraction increases readability.} 
Type aliasing lets us abstract from the actual definition of data types.
When writing our source code, we can define our type aliases before writing the
rest of our program.
%% skriv bedre
If we want to define a sphere in a three dimensional space, we want to define it
with \\
\textbf{1) a radius}, \textbf{2) a position} , and \textbf{3) a direction} that it is
moving.\\
Let us define a function that multiplies the speed of the sphere by a factor k:
\begin{lstlisting}
fun {f32, {f32, f32, f32}, {f32, f32, f32}} multiply_velocity(
	{f32 r, 
	{f32 x_pos , f32 y_pos , f32 z_pos} , 
	{f32 x_dir,f32 y_dir, f32 z_dir}} , f32 k) 
	  =
    {r, {x_pos, y_pos, z_pos}, {k * x_dir , k * y_dir, k * z_dir }}
\end{lstlisting}
\noindent
With type aliasing, we can compartmentalize the data type, and remove the need
for explicitally typing out every parameter of function input.
Coupled with helper functions, we can now multiply the speed of the sphere like
this:
\begin{figure}
\begin{verbatim}
type vec3 = {f32, f32, f32}
type position, direction = vec3 
type radius = f32
type sphere = {radius, position, direction}

fun vec3 multiply_vector(vec3 {pos_x, pos_y, pos_z}, f32 k) =
  {k * pos_x , k * pos_y , k * pos_z}

fun sphere multiply_velocity(
  sphere {radius, position, direction}, f32 factor)  =
    let new_direction = multiply_vector(direction, factor)
    in {radius, position, new_direction}
\end{verbatim}
\end{figure}

\noindent As we are now using the sphere type as the function argument, we can pattern
match on the type aliased values contained in the sphere type.
Most importantly, the vectors of the sphere are abstracized into a single variables
instead of tuples.
\\
The end result is a shorter, more readable program~\ref{nbody_after}

\subsubsection{Compartmentalization of functionality increases usability:}
Splitting code functionality into multiple files, will allow the programmer to
compile and type check these modules individually. The programmer can edit and
contribute to these files independantly of the programs which includes these
modules.

\subsubsection{Approximating higher order functionality whilst keeping performance}
\label{subsec:higherorderfunctionality}
It is possible to express higher-order functionality in Futhark, without taking
a performance hit in the compiled Futhark code.
%% reference to slow higher order functionality
We will reiterate on the concept of modules\ref{sec:structures}, by introducing
the concept of functors.\\
First we repeat the three-dimensional vector module
from earlier, but without declaring any particular
primitive \ref{futharks_types} type as the contained type of the vector:
\begin{verbatim}
  struct Vec3 {
      type vector = {t, t, t}
      fun vector add( ... ) = ...
      fun vector subtract( ... ) = ...
      fun vector multiply ( ... ) = ...
      fun vector divide ( ... ) = ...
  }
\end{verbatim}
\noindent
The structure above cannot be used on its own. Type \texttt{t} is not
instantiated, and the module cannot be type checked, which causes an error.

\noindent
We can solve that problem, by instantiating the abstract structure, using a
simple functor; the \texttt{where}-clause.
\label{first_functor_example}
\begin{verbatim}
  struct Int {
    type t = int 
  }
  struct IntVec3 = Vec3(Int)
\end{verbatim}
\noindent
We can now access the structure \texttt{IntVec3} throughout the rest of the
program.
\\
The structure \texttt{IntVec3} is \texttt{Vec3}, except all instances
of type \texttt{t} in \texttt{Vec3} is exchanged with type \texttt{int}.
\\
\\
\textbf{To recap:} functors allows us to define an abstract implementation of some
structure \textbf{ONCE}, and lets us instantiate this structure any number of
times, each time with our own type.
\\
\\
From a performance-concerned point of view, the module system is desirable, as the Futhark compiler handles all used and included structures at compile time.
\subsection{Problem definition}
\label{subsec:probdef}
Is it possible to implement a module system in Futhark, which displays features
comparable to the module system implemented in Standard ML? \cite{crash_course}.

\subsubsection{Scope of project}
\label{subsec:project_scope}
The scope of the project is
\begin{itemize}
  \item to define and implement a type aliasing system in Futhark
  \item to define and implement a module system in Futhark, which has:
    \begin{itemize}
      \item The definition of structures containing types and functions
      \item Nested modules; meaning that any structure can contain a structure
      \item A well defined way of referring to types, structures and functions
        contained in a structure
    \end{itemize}
  \item to research the possibility of implementing functor functionality, so that Futhark supports the
    definition of abstract structures and concretizations of these structures.
    \\    A suggested design for the implementation of functors should be part of this project.
\end{itemize}
\clearpage

\subsection{Related work}
Futhark modules are designed with Standard MLs module system in mind. Thus, \textit{The
Definition of Standard ML '97} \ref{definition} has been a source of inspiration
for the behaviour of the Futhark module system. \\
However, I have not read the code implementation of SML modules. My code supervisor and I did not find it necessary to lean on the SML implementation of modules:\\
The Futhark compiler could readily support extensions to the language, and it was not difficult to
figure out a way to implement the module system.
\\
\\
Implementing SML modules in other languages is not a unique idea:\\
Amin\ref{scala_modules}Yaraw has shown a way of implementing SML modules in Scala
language. However, he was able to exploit functionality that was already available in
Scala, to create his module system.
\\
\\
Mainly, he was able to use Scala \texttt{functions} as functors to create new Scala
\texttt{classes} (comparable to structures), just as Scala
\texttt{traits} were used for implementing signatures.
\\
\\
Futhark had neither functions (for structures), structures or signatures at the
beginning of the project, which made these necessary to add to the compiler itself.
\clearpage
\section{Type aliases}
To make an implementation of functors\ref{functors} in Futhark, it was necessary
to implement type aliases first.



\subsection{The language}\label{futharks_types}
The initial type system in Futhark supported the following type definitions:
\begin{align*}
  Type           & = & \texttt{Primitive type} \\
                 &\ | & ( Types ) \\
                 &\ | & [ Type ] \\
  \\
  \\
  Types          & = & Type\ ,\ Types \\
                 &\ | & Type \\
  \\
  \\
  Primitive type & = & UnsignedInteger \\
                 &\ | & SignedInteger \\
                 &\ | & FloatType \\
                 &\ | & Boolean \\
\end{align*}

Implementing type aliases expands our type constructions as following:
\begin{align*}
  Type           & : & \texttt{Primitive type} \\
                 &\ | & ( Types ) \\
                 &\ | & [ Type ] \\
                 &\ | & TypeAlias \\
  \\
  \\
  Types          & : & Type\ ,\ Types \\
                 &\ | & Type \\
  \\
  \\
  Primitive type & : & \texttt{UnsignedInteger} \\
                 &\ | & \texttt{SignedInteger} \\
                 &\ | & \texttt{FloatType} \\
                 &\ | & \texttt{Boolean} \\
  \\
  \\
  TypeAlias      & :  & \texttt{type}\ strid\ \texttt{=}\ Type \\
\end{align*}
where $type\_id$ is short for \textit{string id}, a string identifier.

\subsection{Interference rules}
A type alias in a Futhark program is done like this: \texttt{type} \textit{Strids} \texttt{=} \textit{Type} ,\\
where \textit{Type} is as defined in the grammar above.
\\
\label{typealiasinterference}
\begin{tcolorbox}
Given an environment
$\Gamma : \{FE, TE\},\ Types = \{ strid \to Type \},\ Type \to \tau\footnote{We will let $\tau$ designate a primitive type in Futhark}$
we can define the following interference rules for using type aliases:

\begin{prooftree}
  \AxiomC{$\Gamma \vdash typedecl \Rightarrow \Gamma ' $}
  \AxiomC{$\Gamma \vdash Type \to \tau$}
  \AxiomC{$\Gamma \not \vdash type\_id \to Type$}
    \TrinaryInfC{$\Gamma \vdash$ \texttt{type} \textit{type\_id }\texttt{=}
      \textit{Type} $\Rightarrow$ \{$ \emptyset , \{ type\_id \to Type \}\}
      \oplus \Gamma$}
\end{prooftree}
where
\begin{align*}
             & \{ \emptyset , \{ strid \to Type \} \} \oplus \Gamma  \\
 \Rightarrow & Types(\Gamma) \leftarrow \ Types(\Gamma) \cup \{strid \to Type\}
\end{align*}
\\
iff the type alias declaration does not
clash with the current environment. There is no clash, if the declaration obeys
the three rules in the implementation subsection\ref{typealiasimplementation}.
\\
\\
We can assign the same type to several string ids simultaneously, as long as we don't declare the same type alias twice.
\begin{prooftree}
  \AxiomC{$\Gamma \vdash typedecl \Rightarrow \Gamma ' $}
  \AxiomC{Strids = $strid_1 , ... , strid_n$ }
    \BinaryInfC{ $\Gamma \vdash$\texttt{type }\textit{Strids }\texttt{= }
      \textit{Type} $\Rightarrow$ $\{ strid_1 \to Type, .. , strid_n \to Type \} \oplus \Gamma \Rightarrow \Gamma' \}$}
\end{prooftree}
      where
\begin{align*}
             & \{ strid_1 \to Type, \cdots,  strid_n \to Type, \} \oplus \Gamma  \\
 \Rightarrow & Types(\Gamma) \rightarrow \ Types(\Gamma) \cup \{ strid_1 \to Type, \cdots,  strid_n \to Type \}
\end{align*}
      

\end{tcolorbox}
\subsection{Implementation}\label{typealiasimplementation}
A type alias $strid_i \to Type$ declaration is succesful, if three rules are followed:
\begin{enumerate}
  \item The alias $strid_i$ is not already declared\footnote{This is by convention of
      only being able to define a value \textit{once}} in the current local
    environment. I.e. the example below:
    \begin{verbatim}
      type foo = i32
      type foo = f32
    \end{verbatim}
\clearpage
  \item The alias $strid_i$ refers to a type (or a type alias), that is either already
    defined in the current environment (including structure environments), or is
    in the same declaration chunk as $strid_i$. In the example below, \texttt{foo} refers
    to \texttt{bar,} but \texttt{bar} is in the same chunk as \texttt{foo}.\\
    \\
    Therefore, \texttt{foo} can be resolved by resolving \texttt{bar}. The implementation of this
    described in a later subsection\ref{typealiasingcode}.
    \begin{verbatim} 
      type foo = bar
      type bar = {f32, i32}
    \end{verbatim}
    There are no hard limit to the number of type aliases that has to be
    checked, before a type alias is resolved:
    \label{longchain}
      \begin{verbatim}
        type foo = bar
        type bar = {f32, baz, i32}
        type baz = [{bee, bang, boo}]
        type boo ...
  
        ...
  
        type bep = i32
        \end{verbatim}
    Such a chain of type aliases is allowed, as long as the last of the three
    rules is obeyed:
    \\
    \\
  \item The alias being resolved cannot be cyclically defined.\label{cyclicaldefinitionerror}\\
    Imagine that some type \texttt{type my\_type = foo} is in the chain in the
    type aliasing example\ref{longchain} above.\\

    In this case, the compiler is first trying to resolve \texttt{foo} by
    resolving \texttt{bar,} and trying to resolving \texttt{bar} by resolving \texttt{baz}, et cetera. \\
    At some point, the compiler encounters \texttt{my\_type}, and must resolve \texttt{foo} to
    continue - which creates a cycle, because \texttt{foo} is resolved by \texttt{bar}, and so on.

    To keep track of this, the compiler maintains the set of aliases that has
    been visited in the attempt to resolve some type alias. Every time another
    type alias has to be checked, the compiler first checks the set to find out,
    whether this alias is already on the list of aliases that needs to be
    resolved.

    If so, the compiler returns an error.
    The implementation of this can be read here\ref{typealiasimplementation}
\end{enumerate}
\subsection{Parsing a type alias}
\subsubsection{Data types for describing a type}
Initially, the types parsed in a Futhark program were always represented as instances of the
following datatype \texttt{TypeBase}:
\begin{lstlisting}[language=Haskell]

  data TypeBase shape as vn = Prim PrimType
                            | Array (ArrayTypeBase shape as vn)
                            | Tuple [TypeBase shape as vn]

  data ArrayTypeBase shape as vn =
        PrimArray PrimType (shape vn) Uniqueness (as vn)
      -- ^ An array whose elements are primitive types.
      | TupleArray [TupleArrayElemTypeBase shape as vn] (shape vn) Uniqueness
      -- ^ An array whose elements are tuples.

  data TupleArrayElemTypeBase shape as vn =
    PrimArrayElem PrimType (as vn) Uniqueness
  | ArrayArrayElem (ArrayTypeBase shape as vn)
  | TupleArrayElem [TupleArrayElemTypeBase shape as vn]
\end{lstlisting}
\noindent
These are also the data types that are used in the internal Futhark program.
\\
Before this project, Futhark parsed types from Futhark source code directly to
TypeBases.
However, we decided to add an intermediate data type between raw source
code and TypeBases, to make type aliases available.

\begin{lstlisting}[language=Haskell]
data UserType vn = UserPrim PrimType SrcLoc
                 | UserArray (UserType vn) (DimDecl vn) SrcLoc
                 | UserTuple [UserType vn] SrcLoc
                 | UserTypeAlias LongName SrcLoc
                 | UserUnique (UserType vn) SrcLoc

    deriving (Show)
\end{lstlisting}
\noindent
The parser was changed, so type declarations in Futhark source would now be parsed as UserTypes and not
typebases.
\pagebreak
\subsubsection{Adding resolved types to scope}\label{typealiasingcode}
\begin{lstlisting}[language=Haskell, numbers=left]
  type TypeAliasTableM =
  ReaderT (HS.HashSet LongName) (StateT Scope TypeM)

typeAliasTableFromProg :: [TypeDefBase NoInfo VName]
                       -> Scope
                       -> TypeM Scope
typeAliasTableFromProg defs scope = do
  checkForDuplicateTypes defs
  execStateT (runReaderT (mapM_ process defs) mempty) scope
  where
        findDefByName name = find ((==name) . typeAlias) defs

        process :: TypeDefBase NoInfo VName
                -> TypeAliasTableM (StructTypeBase VName)
        process (TypeDef name (TypeDecl ut NoInfo) _) = do
          t <- expandType typeOfName ut
          modify $ (addType name t)
          return t

        typeOfName :: LongName -> SrcLoc
                   -> TypeAliasTableM (StructTypeBase VName)
        typeOfName (prefixes, name) loc = do
          inside <- ask
          known <- get
          case typeFromScope (prefixes, name) known of
            Just t -> return t
            Nothing
              | (prefixes, name) `HS.member` inside ->
                  throwError $ CyclicalTypeDefinition loc name
              | Just def <- findDefByName name ->
                  local (HS.insert (prefixes, name)) $ process def
              | otherwise ->
                  throwError $ UndefinedAlias loc name
\end{lstlisting}
\noindent
TypeAliasTableM is a monad stack that is used to resolve a list of type alias
declarations in a declaration chunk. \\
It is a reader monad transformer that contains a state monad transformer, that
contains the TypeM monad.
\\
\\
\noindent
The reader monad is used to contain a HashSet as its environment. This
environment is used to keep check of cyclical definitions as described in
\ref{cyclicaldefinitionerror}.
\\
\\
For each type alias, we use the function \texttt{process} to resolve the type, and modify
the scope contained within the state monad of the ReaderT.
\\
\\
Resolving a type from a type aliasing is done using the function \texttt{typeOfName} in
the code snippet\ref{typealiasingcode}. \texttt{typeOfName} tries to
resolve the type by name by retrieving the scope contained in the transformed
State monad inside the reader.
\clearpage
\noindent
If this is not immediately possible, we must either continue our search for the type, throw an error due to a cyclical type definition, or throw an error because the type alias has not been defined yet.
\\
If our attempt to resolve $strid_a$ leads to another alias
$strid_b$\footnote{line 27-33}, we add
$strid_a$ to our reader environment using the function \texttt{local}, and
process $alias_b$ instead.
\\
\\
\begin{tcolorbox}
\textbf{Acknowledgement}: The initial design of \texttt{expandType} and the addition of
type aliases to the scope was initially much larger, but was reduced in size by
Troels Henriksen, who rewrote the process to use monads, and reduced some code
duplication. 
\end{tcolorbox}
\subsubsection{Converting UserType to TypeBase}
\begin{lstlisting}[language=Haskell, numbers=left]
expandType :: (Applicative m, MonadError TypeError m) =>
               (LongName -> SrcLoc -> m (StructTypeBase VName))
            -> UserType VName
            -> m (StructTypeBase VName)

expandType look (UserTypeAlias longname loc) =
  look longname loc
expandType _ (UserPrim prim _) =
  return $ Prim prim
expandType look (UserTuple ts _) =
  Tuple <$> mapM (expandType look) ts
expandType look (UserArray t d _) = do
  t' <- expandType look t
  return $ arrayOf t' (ShapeDecl [d]) Nonunique
expandType look (UserUnique t loc) = do
  t' <- expandType look t
  case t' of
    Array{} -> return $ t' `setUniqueness` Unique
    _       -> throwError $ InvalidUniqueness loc $ toStructural t'

\end{lstlisting}

\subsubsection{Why we added UserType instead of extending TypeBase}
Adding UserType and then resolving these into TypeBases whilst running the
program through the TypeChecker, removes the need of handling UserAliases after
the type check, where these aliases are resolved.
\\\\
\label{typeclarification}Furthermore, not all information about a TypeBase declaration
can actually be claimed already at program parse time.
Some information about i.e. IKKE SIKKER HER array dimensionality in regards to aliased arrays, is decided within the type
checker as well. 

\subsubsection{The slip from type aliases to realized types}
Since every type alias is resolved in the type checker, the UserTypes are not
used after the type check.

The data type for a type declaration is this:
\begin{lstlisting}[language=Haskell]
data TypeDeclBase f vn =
  TypeDecl { declaredType :: UserType vn
                             -- ^ The type declared by the user.
           , expandedType :: f (StructTypeBase vn)
                             -- ^ The type deduced by the type checker.
           }
\end{lstlisting}
An unresolved type looks like this:\\\\
\indent \texttt{TypeDecl userType NoInfo}\\
\\
After resolve, NoInfo has been filled out with a variable of type\\
\texttt{Info TypeBase}, giving us the following TypeDecl:\\\\
\indent \texttt{TypeDecl usertype (Info typebase)}.\\
\\
At any point after the type check, only the \texttt{expandedType} of TypeDecl is used.

\subsection{Results}
The addition of type aliases works without any issues.
To verify this, Futhark has been tested to pass all of the tests in futharks
test suite\footnote{located in folder \texttt{futhark/data/tests}}.

However, it was also necessary to write tests to specifically confirm, that the
implementation respects the rules defined in \ref{typealiasimplementation}

\subsubsection{A type cannot be defined twice in the same environment}
From \texttt{alias-error3.fut} in \texttt{futhark/src/data/tests/types}:
\begin{verbatim}
-- You may not define the same alias twice.
--
-- ==
-- error: Duplicate.*mydup

type mydup = int
type mydup = f32

fun int main(int x) = x
\end{verbatim}

This program fails as expected.

\subsubsection{A type alias cannot be defined, if it refers to a type alias that has not been defined}
From \texttt{alias-error4.fut} in \texttt{futhark/src/data/tests/types}:
\begin{verbatim}
-- No undefined types!
--
-- ==
-- error: .*not defined.*

type foo = bar

fun foo main(foo x) = x
\end{verbatim}
This program fails as expected.

\subsubsection{A type alias cannot be cyclically defined}
From \texttt{alias-error5.fut} in \texttt{futhark/src/data/tests/types}:
\begin{verbatim}
-- No tricky circular types!
--
-- ==
-- error: .*cycl.*

type t0 = [t1]
type t1 = {int, float, t2}
type t2 = t0

fun t1 main(t1 x) = x
\end{verbatim}
This program fails as expected.
\clearpage
\subsubsection{Example of planet simulations being simplified by type aliases}\label{nbody}
A nice example of the benefits of type aliasing, is the N-body simulation
(\texttt{nbody}), which is a simulation over the n-body
problem\footnote{\url{https://en.wikipedia.org/wiki/N-body\_problem}}.

The original Futhark implementation of the simulation contained function
arguments of tuples with arity up to 10. Whilst it is still necessary to bring
all the arguments throughout the program, type aliasing makes the program itself
much more readable:
\subsection{Future work}
To make functors work, it must be possible to declare an abstract type alias in
a structure.\\
\\
That will allow for a structure definition, where a type variable\footnote{in the form of a \textit{strid}} has been declared, but not defined, until the containing structure is instantiated using a functor.
\\
This has not been implemented yet.
\clearpage
\clearpage
\section{Structures}
\label{sec:structures}
\textbf{Introduction to structures:}\\
\label{subsec:structuresaregood}
Without structures, every function and type in futhark shares the same scope.
Implementing modules lets us create functions that are alike, but keeps
distinctions between them.
Take this example of a program with two different vector types:

\begin{verbatim}
type vec3 = {f32, f32, f32}
type vec4 = {f32, f32, f32, f32}

fun vec3 vec3_plus(
  vec3 {a_1, ... , a_3}, 
  vec3 {b_1, ..., b_3}
  ) = {a_1 + b_1, ... , a_3 + b_3}

fun vec4 vec4_plus(
  vec4 {a_1, ... , a_4}, 
  vec4 {b_1, ..., b_4}
  ) = {a_1 + b_1, ... , a_4 + b_4}
\end{verbatim}
\noindent
Let us try compartmentalizing a vector type and its functions into a structure.\\
  In the following example, we have defined two different modules, each
containing a structure,
and a futhark program which includes and utilizes these modules:
\begin{verbatim}
Vec3Float.fut:
  structure Vec3Float = 
    struct
      type vector = {f32, f32, f32}
      fun vector plus( ... ) = ...
      fun vector minus( ... ) = ...
      fun vector multiply ( ... ) = ...
    end

Vec4Float.fut:
  structure Vec4Float = 
    struct
      type vector = {f32, f32, f32, f32}
      fun vector plus( ... ) = ...
      fun vector minus( ... ) = ...
      fun vector multiply ( ... ) = ...
    end
\end{verbatim}
\clearpage
\begin{verbatim}
myprogram.fut:

  include Vec3Float
  include Vec4Float

  type vec3 = Vec3Float.vector
  type vec4 = Vec4Float.vector
  
  fun vec4 foo(vec3 vector) = 
    let {a, b, c} = Vec3.plus(vector, vector)
    in Vec4.multiply({a, b, c, 1.0f} , 4.0f)
\end{verbatim}
\noindent
Whilst it \textit{is} possible to create libraries without a module
implementation \footnote{By including libraries that adds functions to the top
  level environment}, the user runs a risk of running into errors like
\texttt{MulipleDefinitionError}\footnote{Multiple functions of same name
  defined}, if any of the library functions uses any of the names, that the
local user is using as well.
\\\\
The module system removes this hazard, as application of functions and types are
done using \textbf{longnames}, which adds prefixes to names. This way, functions
can have the same name, as long as they do not share the same prefix.

\begin{tcolorbox}
\textbf{Longnames:}\\
A longname consists of any amount of prefixes followed by a dot, followed by the
string id of the desired function or type:
\begin{align*}
LongName & :  & \texttt{prefix.}LongName \\
         &\ | & \texttt{identifier}
\end{align*}

In Futhark we will be using string ids for declarations, and longnames for
the accessing types and functions in structures.


\end{tcolorbox}
\subsubsection{Accessing types and functions within structures}
To work with type aliases and modules, we need to define the internal
environment of Futhark during compile time.\\
\\
Before starting this project, the environment of Futhark could be described like
this:\\
$\Gamma = (FE)$ , where \textit{FE} is a function environment, mapping function ids to functions:

 $\{funid \to funexpr\}$.
\\\\
It is the goal of this project to expand the environment of Futhark, so that\\[0.2em]
$\Gamma = (FE, TE, SIGE, STRUCTE, FUNCTE)$, where the additions are a type alias environment, a signature environment, a structure environment and a functor environment.\\[0.2em]
Type aliases, signatures, structures and functors are described in their respective
sections.\\
A structure can be regarded as a structure name and an environment contained in the structure, so that
$\{strid \to \Gamma_{strid}\}$ is a mapping in \textit{STRUCTE}.
\clearpage
Any $\Gamma_{strid}$ contains its own function-, typealias-, signature-, structure- and functor declarations.\\
\\

\subsection{Interference rules}\label{structuresinterferencerules}
Now that we can discern between the global environment, and any number of
environments in structures, we must redefine the environment behaviour of Futhark at compile time.
\\
\begin{figure}\label{Rule1}
  \begin{tcolorbox}
    \begin{prooftree}
      \AxiomC{$\Gamma \vdash fundecl \Rightarrow \Gamma'$}
      \AxiomC{$\Gamma \not\vdash fun\_id \to funexpr$}
      \BinaryInfC{
      $\Gamma \vdash \texttt{fun }fun\_id \texttt{ = }funexpr
        \Rightarrow \{
        \{fun\_id \to funexpr\} ,
        \emptyset ,
        \emptyset ,
        \emptyset ,
        \emptyset
        \} \oplus \Gamma$}\\
    \end{prooftree}
    iff the function the function body is otherwise well-formed\footnote{I will not go further into
      the functionality of functions, as it is not within scope of this project.}.
  \end{tcolorbox}
  \caption{Rule for adding a function to the environment}
\end{figure}

After expanding the environment to contain the four new elements, we can define the
addition of any of these four elements follow a generalized rule:
\begin{figure}\label{Rule2generalized}
  \begin{tcolorbox}
    \begin{prooftree}
      \AxiomC{ $\Gamma \vdash Decl_1 \Rightarrow \Gamma' $ }
      \AxiomC{ $\Gamma \not\vdash id \to decltype\ Decl_2$ }
        \BinaryInfC{ $\Gamma \vdash$ \texttt{decltype} \textit{id }\texttt{=} \textit{decl} $\Rightarrow \{ id \to Decl \} $ }
      \end{prooftree}
    \end{tcolorbox}
\end{figure}
Please note, that several declarations can actually share name \textit{id}, as
long as they don't share declaration type.\\
\\
\subsubsection{Interference rules for adding multiple declarations}
Finally, we want to create a rule for which environment we get, when we state
multiple declarations in a row.
\begin{figure}\label{Rule2multiple}
  \begin{tcolorbox}
    \begin{prooftree}
      \AxiomC{ $\Gamma \vdash Decl_1 \Rightarrow \Gamma_1 $ }
      \AxiomC{ $\Gamma \oplus \Gamma_1 \vdash Decl_2 \Rightarrow \Gamma_2 $}
        \BinaryInfC{ $\Gamma \vdash Decl_1 , Decl_2 \Rightarrow \Gamma_1 \oplus \Gamma_2$}
      \end{prooftree}
      For any $\Gamma$, $\Gamma_a \oplus \Gamma_b \Rightarrow \Gamma_a \cup \Gamma_b$
    \end{tcolorbox}
\end{figure}
This rule is covers the run of an entire program, since we can set the initial
declaration in the program as $Decl_1$, use rule \ref{Rule2multiple} to handle
the first declaration in the pair $Decl_1$ and $Decl_2$, and afterwards using
the same rule again on $Decl_2$, $Decl_3$.
\\
This procedure is repeated until the rule reaches the two last statements,
$Decl_{n-1}$, $Decl_n$. 
We must take a closer look at what happens, when a structure is defined.
The definition of a structure in Futhark has the following behaviour.
\subsubsection{Rule for adding a structure to the local environment}
\begin{figure}[h!]\label{Rule3}
  \begin{tcolorbox}
    \begin{prooftree}
      \AxiomC{ $\Gamma \vdash strdecl \Rightarrow \Gamma'$ }
      \AxiomC{ $\Gamma \vdash strdecls \Rightarrow \Gamma_{strdecls}$}
      \AxiomC{ $\Gamma \oplus \Gamma_{strdecls} \vdash \Gamma_{strid}$}
        \TrinaryInfC{$\Gamma \vdash$ \texttt{struct} \textit{strid}
          \texttt{\{} \textit{strdecls} \texttt{\}}
          $\Rightarrow \{ strid \to \Gamma_{strid}\}$}
    \end{prooftree}
    where $\Gamma \oplus \Gamma_{strdecls} \Rightarrow \Gamma \cup
    \Gamma_{strdecls}.$
    If there is a shared identifier in both $\Gamma$ and $\Gamma_{strdecls}$,
    the definition in $\Gamma_{strdecls}$ is the definition in $\Gamma_{strid}$.
  \end{tcolorbox}
\end{figure}
The consequence of defining $\Gamma_{strid}$ as a union of new declarations together with the old environment menas, that Futhark exhibits variable shadowing, where definitions in a structure are valid in its nested structures, unless they are redefined in the nested structure.

\subsubsection{Interference rules for interpreting functions and types in an
  environment with structures}\label{interpretingfunctionsandtypeswithstructures}
There are three cases where it is necessary to resolve a function or a type from a
longname:\\
1) When applying a function as an expression in a function expression; i.e.
\begin{verbatim}
  let myNumber = Constants.numberFour()
\end{verbatim}
2) When using a function as an argument in a currying function; i.e.
\begin{verbatim}
  let numbers = [1, 2, 3, 4] in
  let sum = reduce(MathLib.plus , 0 , numbers)
\end{verbatim}
3) When using a type definition from a structure; i.e.
\begin{verbatim}
  type int_pair = Pairs.Int.t 
\end{verbatim}
We can define the interference rule for using a longname in the three different
cases. As all three cases handles resolving a longname the same way, the rule
will only be called for the first case:
\begin{figure}\label{Rule5}
  \begin{prooftree}
    \AxiomC{$\Gamma \vdash \mathit{longname} \to fun\_definition$}
    \UnaryInfC{$\Gamma \vdash let x = longname() \Rightarrow x := \downarrow longname$}
  \end{prooftree}
  where
  \begin{align*}
    \downarrow longname & = & return getFromEnv(longname , \Gamma) \\
  \end{align*}.
\end{figure}
\subsection{Implementation}
A program in the Futhark type checker is defined as a list of unchecked
declarations.
I will not describe the process in details, but will instead
explain specific parts of the code.
I will specifically focus on parts of the code, that corresponds to the
interference rules described earlier.
\\
\subsubsection{The Scope datatype}
To describe the environment in the Futhark compiler, we use the Scope datatype:
\begin{verbatim}
data Scope  = Scope { envVtable :: HM.HashMap VName Binding
                    , envFtable :: HM.HashMap Name FunBinding
                    , envTAtable :: HM.HashMap Name TypeBinding
                    , envModTable :: HM.HashMap Name Scope
                    , envBreadcrumb :: LongName
                    }
\end{verbatim}
Given $\Gamma = (Functions, TypeAliases, Structures)$, then \textit{Functions} is
implemented in \texttt{envFtable}, \textit{TypeAliases} in \texttt{envTAtable} and \textit{Structures} in \texttt{envModTable}.
\subsubsection{checkProg}
The type check is initiated in the functions checkProg and checkProg'.
checkProg' checks the initial top level of declaratations for duplicates.
\begin{lstlisting}[language=Haskell]
  checkProg :: UncheckedProg -> Either TypeError (Prog, VNameSource)
  checkProg prog = do
    checkedProg <- runTypeM initialScope src $ Prog <$> checkProg' (progDecs prog')
    return $ flattenProgFunctions checkedProg
    where
      (prog', src) = tagProg' blankNameSource prog
\\  
  checkProg' :: [DecBase NoInfo VName] -> TypeM [DecBase Info VName]
  checkProg' decs = do
    _ <- checkForDuplicateDecs decs
    (_, decs') <- checkDecs decs
    return decs'
\end{lstlisting}
%$
\subsubsection{Checking for duplicates}
\label{subsec:checkingforduplicates}
As described in rule \ref{Rule4}, we allow multiple declarations of same name,
as long as they don't share type:
In example, checking for two function definitions of the same type is done in
the first case of checkForDuplicateDecs:
\begin{lstlisting}[language=Haskell]
checkForDuplicateDecs :: [DecBase NoInfo VName] -> TypeM ()
checkForDuplicateDecs decs = do
  _ <- foldM_ f HM.empty decs
  return ()
  where
    f known dec@(FunOrTypeDec (FunDec (FunDef _ (name,_) _ _ _ loc))) =
      case HM.lookup name known of
        Just dec'@(FunOrTypeDec (FunDec FunDef{})) ->
          throwError $ DupDefinitionError name loc $ decLoc dec'
        _ -> return $ HM.insert name dec known
\end{lstlisting}
      %$
\subsubsection{Dividing a Futhark program into chunks}
To facilitate variable shadowing, it was necessary to split a parsed Futhark
program into chunks.
\\
A structure declaration alters the enviroment drastically, by enabling the
following program declarations following the structure to access the environment
of the structure.
Therefore we divide any list of declaration into type- and function
declarations, and structure declarations.
An example is given in figure\ref{chunks}
\begin{figure}\label{chunks}
  \begin{tcolorbox}
    \begin{lstlisting}[language=Haskell]
      fun int four() = 4
\\
      structure M0 =
        struct
          fun int double(int a) = a + a
        end
\\
      fun int main() =
        let x = four() in
        M0.double(x, x)
  \end{lstlisting}
  \texttt{four()} does not have \texttt{M0} available in its local environment, but
    \texttt{main()} does, because \texttt{M0} HAS already been parsed, at the point where
    \texttt{main()} is declared
  \end{tcolorbox}
\end{figure}
\\
This chunking is done recursively on a list of declarations:
\begin{lstlisting}[language=Haskell]
chompDecs :: [DecBase NoInfo VName]
          -> ([FunOrTypeDecBase NoInfo VName], [DecBase NoInfo VName])
chompDecs decs = f ([], decs)
  where f (foo , FunOrTypeDec dec : xs ) = f (dec:foo , xs)
        f (foo , bar) = (foo, bar)
\end{lstlisting}
\subsubsection{Checking function- and type declarations}
\label{subsec:checkfunortype}
\begin{lstlisting}[language=Haskell, numbers=left]
checkDecs :: [DecBase NoInfo VName] -> TypeM (Scope, [DecBase Info VName])
checkDecs decs = do
\\
    let (funOrTypeDecs, rest) = chompDecs decs
    scopeFromFunOrTypeDecs <- buildScopeFromDecs funOrTypeDecs
    local (const scopeFromFunOrTypeDecs) $ do
      checkedeDecs <- checkFunOrTypeDec funOrTypeDecs
      (scope, rest') <- checkDecs rest
      return (scope , checkedeDecs ++ rest')
\\
checkDecs [] = do
  scope <- ask
  return (scope, [])
\end{lstlisting}
% $
1) The current chunk of declarations is built using chompDecs. \\
2) A scope is built from this chunk, by using the three interference rules in
section \ref{normaldecls}.\\
3) The scope from step 2 is now the local scope in the following function
execution.\\
4) checkFunOrTypeDec does the actual typechecking of the function declaration. \\
5) The remainder of the declarations chunked in chompDecs is checked using
checkDecs. Note, that the first element of \texttt{rest} must be a structure
declaration, due to the implementation of chompDecs.
\subsubsection{Checking structure declarations}\label{checkingstructuredeclarations}
Please note that structures are currently called modules inside Futharks compiler.
\begin{lstlisting}[language=Haskell]
checkDecs (ModDec modd:rest) = do
  (modscope, modd') <- checkMod modd
  local (addModule modscope) $
    do
      (scope, rest') <- checkDecs rest
      return (scope, ModDec modd' : rest' )
\end{lstlisting}
%$
When checkDecs encounters a \texttt{ModDec}, \texttt{checkMod} is called to resolve the \texttt{ModDec}.
The ModDec defines an environment of functions and type declarations called
modscope, which we add to the local environment by calling \texttt{local
  (addModule modscope)}\\
This part of the code is the implementation of rule\ref{Rule3}.
\begin{lstlisting}[language=Haskell]
checkMod :: ModDefBase NoInfo VName -> TypeM (Scope , ModDefBase Info VName)
checkMod (ModDef name decs loc) =
  local (`addBreadcrumb` name) $ do
    _ <- checkForDuplicateDecs decs
    (scope, decs') <- checkDecs decs
    return (scope, ModDef name decs' loc)
\end{lstlisting}
%$
checkMod first adds a ``breadcrumb'' of the structure's name (\textit{strid}) to
the local environment. This is where the transformation in rule \ref{Rule3} of
$\Gamma_{strdecls} \Rightarrow \Gamma_{strid}$ is implemented.
\\
After the environment has been given its name, the internal declaration of the
structure is read using \texttt{checkDecs}.
\\
\subsubsection{Resolving the application of a longname}
Resolving the application for a longname is done through two short recursive
functions.
\\
Figure \ref{resolvelongname} shows the implementation of the longname rule
\ref{Rule5}:
\begin{figure}\label{resolvelongname}
  \begin{lstlisting}[language=Haskell]
    type LongName = ([Name], name)

    typeFromScope :: LongName -> Scope -> Maybe TypeBinding
    typeFromScope (prefixes, name) scope = do
      scope' <- envFromScope prefixes scope
      let taTable = envTAtable scope'
      HM.lookup name taTable
    
    funFromScope :: LongName -> Scope -> Maybe FunBinding
    funFromScope (prefixes, name) scope = do
      scope' <- envFromScope prefixes scope
      let taTable = envFtable scope'
      HM.lookup name taTable
    
    
    envFromScope :: [Name] -> Scope -> Maybe Scope
      envFromScope (x:xs) scope =
        case HM.lookup x (envModTable scope) of
          Just scope' -> envFromScope xs scope'
          Nothing -> Nothing
    envFromScope [] scope = Just scope
\end{lstlisting}
\end{figure}
%$

\subsubsection{Including structures, functions and types from other files}\label{structuresincludes}
The Futhark \texttt{include}-statement lets the user include
other files into the current program.

This is implemented by letting the Futhark compiler combine the source code from
all the included files, with the declarations in the main program. The combined
program is then sent passed on through the rest of the compiler.

As the Futhark program with an arbitrary number of includes is merged into a
single program within the compiler, we can compartmentalize a program into
discrete files.

However, this creates a hazard: there might be declarations in the included
code, which has names that overlap with names in the remaining code, so that a
\texttt{DuplicateDefinitionError} is triggered. the included code has any declarations
that results in a duplicate definition error. Refer to
\ref{structuresfuturework} for a possible future solution.

\subsubsection{Keeping track of function names}
Futharks variable shadowing made it necessary to extend the type of the function
declaration type. \\
Initially, a function had only the declared name of the function as an identifier.
However, it was necessary to extend this name into a pair:

\texttt{type FunName = (declaredName :: Name, expandedName :: LongName)}.

When a function is added to the function table during \texttt{buildFtable}, it
is added to the function table together with it's longname. The function's
longname contains the function's name, and the name of the (potentially) nested
structures it was defined in.

\subsection{Tests}
\label{subsec:structuretests}
To verify that the implementation of structures was correct, a series of test
programs were written to verify, whether the interference rules defined for the
structures, actually held in an executed Futhark program.

Having writting the structure implementation myself, I have been able to
consider my testing options. I have had the choice of writing either \textit{white box}
tests, \textit{black box }tests, or both.
\\
\\
In the context of these futhark structures, white box tests are defined as tests
that are designed by a programmer, who has intricate knowledge of the
code implementation of the program features, he is supposed to test.\\
I.e., this means trying to write programs, which tests whether there can be
ambiguities in the parsing of the program, or other general errors in the code.
\\
\\
Black box tests are written without knowledge of the code behind the program.
In the context of Futhark structures, this limits the test writer to test,
whether the rules which are specified in \ref{structuresinterferencerules} holds
in the actual implementation.

In the following tests, programs which break the rules are expected to return
with an error.

\subsubsection{Testing rule for multiple declarations [...]}
The following tests are implemented to test whether rule \ref{Rule4} holds.
\texttt{duplicate\_def0.fut}:
\lstinputlisting[language=Haskell]{./tests/duplicate_def0.fut}
This test passes.
\\
In the following test, the structure foo contains declarations of name foo also.
This is a white box test to ensure, that the implementation handles type, struct
and fun declarations in seperate tables.
\texttt{duplicate\_def1.fut}:
\lstinputlisting[language=Haskell]{./tests/duplicate_def1.fut}
This test passes.
\\
In the following tests, the programmer has attempted to declare functions and
types in the same environment, twice.

\texttt{duplicate\_error0.fut}:
\lstinputlisting[language=Haskell]{./tests/duplicate_error0.fut}
This test passes.
\\
In the following test, the structure foo contains declarations of name foo also.
\texttt{duplicate\_error1.fut}:
\lstinputlisting[language=Haskell]{./tests/duplicate_error1.fut}
This test passes.

\subsubsection{Testing structures can be called as expected}
The following tests are implemented to test that calling structures works as
defined in rule \ref{Rule5}.

\texttt{calling\_nested\_module.fut}:
\lstinputlisting[language=Haskell]{./tests/calling_nested_module.fut}
This test passes.
\\
Currying functions such as map and reduce works as expected:\\
\texttt{map\_with\_structure0.fut}:
\lstinputlisting[language=Haskell]{./tests/map_with_structure0.fut}
This test passes.
\\
Using structures from an include works as expected:\\
\texttt{Vec3.fut}:
\lstinputlisting[language=Haskell]{./tests/Vec3.fut}
\lstinputlisting[language=Haskell]{./tests/triangles.fut}
This test passes.
\subsubsection{Testing rule for variable shadowing}
The following tests are implemented to test whether the rule \ref{Rule4a} holds:
Simple shadowing for functions holds:\\
\texttt{shadowing0.fut}:
\lstinputlisting[language=Haskell]{./tests/shadowing0.fut}
Simple shadowing for types holds:\\
\texttt{shadowing1.fut}:
\lstinputlisting[language=Haskell]{./tests/shadowing1.fut}
This test shows, that structures are only read into scope, after they are fully parsed.
\texttt{shadowing2.fut}:
\lstinputlisting[language=Haskell]{./tests/shadowing2.fut}
and
\texttt{undefined\_structure\_err0.fut}:
\lstinputlisting[language=Haskell]{./tests/undefined_structure_err0.fut}

\subsection{Results}
\label{subsec:structuresresults}
The implementation of structures works, and I am satisfied with the results. As
shown in the tests, the structures behave as prescribed in the rules\ref{structuresinterferencerules}.

To answer the problem definition, it is definitely possible to add module
functionality to Futhark, in terms of structure support.

The extensions to the Futhark compiler in the TypeChecker.hs-module have had perfect compatibility with the rest
of the Futhark, as all of the type aliases, structures and so on, are dealt with
in the TypeChecker, which returns a typechecked Futhark program to the Futhark
internaliser.

Besides two small fixes in Internalise.hs, it has not been necessary to change
any part of the Futhark compiler pipeline, after the TypeChecker step.
\subsection{Future work}
\label{subsec:structuresfuturework}
As mentioned in \ref{structureincludes}, it is possible to trigger a
\texttt{DuplicateDefinitionError} from including files into the main program.

Given more time, I would like to extend the Futhark \texttt{include} statement
to support an \texttt{as} statement, so that the inclusion

\texttt{include some\_module as M0}

will include the declarations from \texttt{some\_module} into the futhark program, but not
into the top level declarations. Instead, the declarations will be loaded into a
structure \texttt{M0}, which can then be accessed throughout the rest of the program as
any other module.
\\
\\
Another potential effort worth of mentioning is the commence the definition of a
standard library for futhark. Futhark already has built-ins mathematical
functions like log10(float a), so defining a standard library of structures
which define i.e. Integer-functions and Float-functions, could be the next step.
\clearpage
\section{Signatures}
Just like a type signature of a function is the definitions of the argument- and
return types of a function, we can define module signatures similarly:\\\\
A module signature specifies the type signatures for a set of declarations.\\
Let us define a signature called \texttt{hasPlus}:
\begin{lstlisting}
  sig hasPlus {
    type t
    val plus : (t,t) -> t
  }
\end{lstlisting}

Then, after this declaration, we can define some structure, and append this
structure declaration with a signature:
\begin{lstlisting}
  include hasPlus
  
  struct M0 {
    ...
  } : hasPlus
\end{lstlisting}
With this signature, we promise that the structure \texttt{M0} matches the declarations
in signature \texttt{hasPlus}.
In the instance of \texttt{M0}, we now require, that \texttt{M0} has a
declaration, where alias \texttt{t} is assigned.\texttt{M0} must also have a
function, that takes two arguments of type \texttt{t}, and returns a value of
type \texttt{t}.

In the following example, two structures are written. \texttt{M0} adheres to the
specifications in \texttt{hasPlus} , but \texttt{M1} does not, which ultimately results in an
error at compile time:

\begin{lstlisting}
  include hasPlus
  
  struct M0 {
    type t = int
    fun t plus(t a, t b) = a + b
  } : hasPlus
  
  struct M1 {
    type t = (int, int)
    fun int plus(t tuple) =
      let (a, b) = tuple
      in a + b
  } : hasPlus
\end{lstlisting}
\subsection{Purpose of module signatures}
By themselves, the ability to add a signature to a structure does not add any
significicant usefulness to a language. Signatures are useful together with
\textit{functors}(Section ~\ref{sec:functors}), so I will refrain from doing further
explanation outside of functor context.


\subsection{Implementation}
\textit{This section is not complete, as I have not implemented a complete and working
signature system at this point.
}

We need a data type which represents a signature.
I chose to define it as a table from some declaration name to a signature
binding:
\begin{lstlisting}[language=Haskell]
type Signature = HM.HashMap Name SigBinding

data SigBinding = FunSignature (Maybe (TypeDeclBase Info VName)) (TypeDeclBase Info VName)
                  -- ^ A function may take arguments, but might also be a
                  constant.
                  -- It always has a return type.

                | TypeSignature (TypeDeclBase Info VName)
                  -- A type signature must resolve to a type.  
                  
                | ModSignature LongName
                -- A structure of signature sig, can itself be demanded to
                -- contain a structure of some ModSignature
\end{lstlisting}

With the Signature defined, we need to extend the data type of the structure to
accomodate the possibility of adding a signature to a structure:

\begin{lstlisting}[language=Haskell]
data ModDefBase f vn = ModDef { modName :: Name
                              , modDecls :: [DecBase f vn]
                              , modDefLocation  :: SrcLoc
                              , modSignature :: Maybe LongName
                              }
\end{lstlisting}
The parsed structure from the futhark program now has either
\texttt{modSignature Nothing} , or \texttt{modSignature (Just }\textit{signature}\texttt{)}.

We update checkMod from subsection ~\ref{checkingstructuredeclarations} accordingly¸ by
calling an extra function \texttt{verifyMod}:

\begin{lstlisting}[language=Haskell]
checkMod :: ModDefBase NoInfo VName -> TypeM (Scope , ModDefBase Info VName)
checkMod (ModDef name decs loc sig) =
  local (`addBreadcrumb` name) $ do
    checkForDuplicateDecs decs
    (scope, decs') <- checkDecs decs
    verifyMod decs' sig loc
    return (scope, ModDef name decs' loc sig)

verifyMod :: [DecBase Info VName] -> Maybe LongName -> TypeM ()
verifyMod _ Nothing _ = return ()
verifyMod decs (Just sig) loc = do
  sigtable <- asks (sigFromScope sig)
  case sigtable of
    Nothing -> throwError $ UndefinedLongName sig loc
    Just t ->
      foldM_ verifyDec decs $ HM.toList t
\end{lstlisting}
%$

For each entry in the signature table, we call \texttt{verifyDec} on this entry to
verify, that decs contain a declaration that satisfies the signature table.

As I have not implemented \texttt{verifyDec} completely yet, this implementation
is not finished.

\subsection{Results}
As the signature implementation is \textit{incomplete}, I cannot claim that it
is possible to implement module signatures in Futhark, by referring to tests.
\\
However, the current implementation lacks \textbf{only} a correct implementation of
\texttt{verifyDec} before it works, barring some corrections of bugs that might
show up during tests.
\\
The ongoing work on this feature can be followed on the \texttt{signatures}-branch in
the public Futhark GitHub repo.
\\
\\
To answer the question in the problem statement; whether it is possible to implement module signatures in
Futhark: \\
Yes, it is. 
\clearpage
% pøls
\section{Functors}
\label{sec:functors}
A function takes a number of arguments of specified types, and returns a value of some return type.
A \textbf{functor} takes a structure\ref{structures} of a specified signature,
and returns a structure of some signature.

The concept is best explained through an example:
\begin{lstlisting}[language=Haskell]
  --
  --==
  -- input {}
  -- output { 4 }

  sig NumType {
    type t
    val add : (t,t) -> t
    val subtract : (t,t) -> t
    val divide : (t,t) -> t
    val multiply : (t,t) -> t
  }

  struct Int {
    type t = int
    fun t add(t a, t b) = a + b
    fun t subtract(t a, t b) = a - b
    fun t divide(t a, t b) = a / b
    fun t multiply(t a, t b) = a * b
    
  } : NumType

  functor Doubler(number : NumType){
    type t = number.t
    fun t add(t a, t b) = number.add
    fun t double(t a) = add(a,a) 
  }

  struct IntDoubler = Doubler(Int)

  fun int main() = IntDoubler.double(2)
\end{lstlisting}
We first declare a signature NumType, which defines the number type. We declare it so
that there is a type t, that has four standard operations (add, subtract, divide
and multiply.).
\\
\\
We then declare a structure Int, which adheres to the NumType signature:
The add function in Int uses the built-in functionality of the plus operator in
Futhark, but could in principle be defined as whatever the programmer would
like \emph as long as the type checks correctly.

Then, the functor is defined: The functor takes a structure as an argument, and
can then refer to this structure in the declarations of the functor, using
dot-notation.
The functor contains a list of type-, function- and structure declarations, just
like the structure does.
However, none of these declarations has to be defined inside the functor:

Any of the declarations can be loaded from the functor's argument instead!

In the example above, the Doubler functor contains a function called
\texttt{double}. But \texttt{double} is not defined inside the functor, but will
instead become whatever the \texttt{double} function is defined as, inside the \texttt{number}
argument of the functor.
At the application of the Doubler functor, creating IntDoubler, we can
guarantee, that our types still hold. This is because the function definitions
in the functor has an abstract type\footnote{type t}, which is not instantiated
until the functor can get the type definition from the structure argument.

To boot, we can even bind a signature to functor if we want to. In that case, we
will have to create a functor, that returns a structure of signature
\textit{sig}, when it is applied.

\subsection{The reason for functors}
\label{subsec:reasonfunctors}
As explained in the introduction of this report\ref{higherorderfunctionality},
functors allows the programmer to write modules with abstract structures,
defining some functionality.
%% the reason is that WE CAN HAVE ALMOST POLYMORPHISM
Functors does not give us actual polymorphism in our programs, but it
\emph{does} allow us to generalize functionality, when it is not tied to any
specific type traits.

\subsection{Tentative implementation of functors}
\label{subsec:implementing_functors}
Implementing a simple functor which instantiates a structure on application,
should not be very difficult.

On a base level, building a structure from a functor and its argument is a
question of parsing the functor's declaration in the right order.\\
At the time of the functor declaration of some functor
\texttt{functor Funct(}\textit{argument} \texttt{:}
\textit{signature}\texttt{)\{ \textit{Functdecls} \}} \, we do not
know anything about the types of the declarations in \textit{Functdecls}.

We cannot type check this functor yet, so we will instead build a functor,
which contains similar declarations to what the structure contains.
However, we allow for abstract declarations of both types, functions and
structures, such as:
\begin{lstlisting}
  functor S(str : sig){
    type a = str.t
    fun a foo(a left, a right) = str.bar(left,right)
    struct M0 = str.mystruct
  }
\end{lstlisting}
We also allow for function declarations with calls to abstract functions:
\begin{lstlisting}
  functor S(str : sig){
    fun int baz = 2 + str.function()
  }
\end{lstlisting}

Thus we will not handle the functor any further, until it is attempted to be
applied to a structure somewhere in the program. At the application of the
functor on some structure \textit{str}, \textit{str} is assumed to be in scope.
\\\\
Our goal now is to convert the applied functor into a structure. This will
happen during checkDecs.
After applying the functor, we will not have any need
for the functor declaration.
Instead, we need a ModDec in the rest of the program, because the Futhark
internalises the program functions on basis of function declarations, which are
read from the actual function declarations \texttt{FunDec}, which are stored
either in the top level decl list, or in the declaration list contained in a
Futhark structure.

In the following tentative implementation of functors in the Futhark
typechecker, we have added a Functor definition to the Futhark syntax.\\
Furthermore, ModDef have been expanded to allow functor application, checkDecs
have been expanded to allow functor application, and checkFunctor has been added.

\begin{lstlisting}[language=Haskell]
  data FunctorDefBase = FunctorDef { functorName :: Name
                                   , functorArg :: LongName
                                   , functorArgSig :: LongName
                                   , functorDecls :: [FunctorDeclBase f vn]
                                   , functorSig :: Maybe LongName
                                   , functorDefLocation :: SrcLoc
                                   }

  ...
                                   
  checkFunctor :: FunctorDefBase f vn -> Name -> TypeM (Scope, ModDefBase Info VName)
  checkFunctor functor structname =
    argScope <- scopeFromEnv $ functorArg functor    
    verifyScope argScope $ functorArgSig functor
    (local (addModule argScope)) $ do
      decs <- resolveFunctorDecs $ functorDecls functor
      (scope, decs') <- checkDecs decs
      verifyMod decs' sig loc
      return (scope, ModDef structname decs' (functorSig functor)  (functorDefLocation functor))
\end{lstlisting}

\clearpage
\section{Discussion of results}
\label{sec:final_section}

Both the type aliasing system and the basic structure implementation developed
during this project, has been designed around a set of interference rules.
Tests were designed to measure whether these rules were obeyed by the actual
implementation of the type system.
For both type aliasing and structures, the final iterations of program showed
that the implementations \textit{did} work as defined in the interference rules.
\\
\\
As my final implementations of the two first features failed neither any of the
feature-specific tests, nor any test in the Futhark test suite, or any of the
Futhark benchmark programs\ref{futhark_benchmarks}, I \textit{will} claim, that
it is possible to extend the Futhark language, and that I am able to succesfully
add features to the code base.
\\
\\
These results suggests that the two remaining features should be quite possible
to implement as well:\\
Although the module signatures were not implemented in time for this report, all the information
needed to perform a signature check at compile time, is present. A correct
implementation is indeed just a question of writing a function, that verifies
the type signature of a declaration correctly.
\\
The implementation of the functor functionality itself was not completed either,
but a tentative implementation has been described in this report.
As the tentative implementation has been described from a good background knowledge
of the relevant parts of the Futhark implementation, I see no immediate reason,
that this implementation should not work \emph barring a total redesign of
the Futhark implementation, outside of my control.
\\

\subsection{Method}
\label{subsec:method}
The developement of the features implemented during this project has generally
followed three phases.\\
For each of the features, I have initially sat down with both my supervisor and
my ``code supervisor'' to discuss both the behaviour, and the implementation of
the features. After having decided on which features we would like each part of
the project to exhibit, we have shortly discussed a strategy for implementing
this.

After these initial design meetings, I have been implementing the designs in the
Futhark compiler itself. As the extensions I have made to Futhark have been
implemented all the way into the Futhark parser, the step from initial code
changes to a test-driven developement has been slightly long.
\\
In the beginning of this project, it took me several weeks from the first change to the program, until
I had code that would compile again. This was mostly caused by me being very
new to the futhark code base. Over time, this was definitely alleviated by me
becoming much more confident in Futhark developement.
\\
At a certain point in the developement cycle of each feature, the implementation
reached an almost finished state: The code would compile, and the initial
parsing tests for the given feature would compile correctly. At this point, I
could commence actual test driven developement:
\\
\\
My method was now to define tests that corresponded to the interference rules
which were declared for the feature in question. Whilst the code \textit{would}
compile at this stage, I reiterated on the implementation to fix mistakes in the
code.
I.e. type checking would perform correctly in a test program, but the program
would exhibit behaviour that was not defined by an interference rule, nor
desirable in the context of program.
\\
Any number of tests could be written to test a feature implementation, but I
concentrated mainly on white box testing, which are attempts to write tests that
breaks the implementation, or makes the implementation show undefined behaviour.
\\
\\
After completing the implementation of a feature, I would document the addition
to the Futhark language in the Futhark language reference documents, and merge
the git feature branch with master.
\subsubsection{Alternatives to the chosen method}
The implementation of Futhark modules was chosen to be done directly in the
Futhark code base.
However, we considered another option in the beginning of the project.
Futhark modules could have been implemented as a prototype instead, using an
intermediate compiler for the modules specifically.
\\
\\
Instead of writing modules directly in a Futhark source file, the user would
write futhark modules in an intermediate language.
This would then be compiled into a normal Futhark source file, without any structures, modules or type aliases.
\\
The benefits of this method would have been a larger degree of freedom in the
implementation: I would have been able to define my an intermediate language for Futhark
modules, and I could have implemented the intermediate language in a language of
my own choice, other than Haskell.
\\
\\
I decided against this, because seperating the module compiler from the Futhark
compiler would demand continued maintenance on the module compiler, whenever the
Futhark language definition changed.\\
Furthermore, I decided to develop the modules in Futhark itself, so I could
benefit from Troels Henriksens knowledge of the Futhark code base, and get more
comfortable with advanced functional programming concepts, such as monads.

\subsection{Conclusion}
Implementing a simple Standard ML style module system is not only theoretically
possible, but should be within reach of the Futhark project, given a couple of
weeks' more of developement time.
\\
At the time of this report, it has not been finished solely due to time constraints.
\\
\\

\subsection{Acknowledgements}
\label{subsec:acknow}
First and foremost, I would like to thank my supervisor \textit{Martin Elsman}, who have
been helpful with advice, theoretical education in relation to defining
environments, and the interference rules hereof.
\\
\\
I would like to thank \textit{Troels Henriksen}, my ``code supervisor'' on this
project. As Troels is one of, if not \textit{the}, head developer of Futhark, he
has been an invaluable source of sound design advice and code reviews on my Futhark
development.
\\
\\
Finally, I like would like to thank \textit{Niels Gustav Westphal Serup}, who
has been helpful with giving me Haskell advice, at times where I have been stuck
in developement.
\\
\\
nå, hej hej

% abstract
% introduction and motivation
%%% usability
%%% portability
%%% achieving higher order functionalilty using modules/functors
%% inspiration from Standard ML
%% problem definition
%% scope of project


% SCOPE MANIPULATION AND BEHAVIOUR

% type aliases
%% the language
%% rules (SAT-like mapping 
%% implementation
%%% the slip from type aliases to realized types
%% results
%%% example of planet simulations being simplified by type aliases
%% future of type aliasing : type a, b, c = vec3

%module system
%% the language
%% nested modules
%% rules and definitions
%% implementation
%%% the /current/ implementation of the slip from typechecker to Internalised funtable
%% results % do this last

% signatures
%% the language
%% rules and definitions (there are none)
%% tentative implementation

% functors
%% the language
%% simple and elaborate functors
%% rules and definitions
%% tentative implementation

% method
% discussion
% conclusion
% takketaler akkrediteringer whatevs

\subimport{appendices/}{appendix_main.tex}

\bibliographystyle{plain}
\bibliography{litteratur}
\end{document}

