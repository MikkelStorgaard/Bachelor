% pøls
\section{Functors}
\label{sec:functors}
A function takes a number of arguments of specified types, and returns a value of some return type.
A \textbf{functor} takes a structure\ref{structures} of a specified signature,
and returns a structure of some signature.

The concept is best explained through an example:
\begin{lstlisting}[language=Haskell]
  --
  --==
  -- input {}
  -- output { 4 }

  sig NumType {
    type t
    val add : (t,t) -> t
    val subtract : (t,t) -> t
    val divide : (t,t) -> t
    val multiply : (t,t) -> t
  }

  struct Int {
    type t = int
    fun t add(t a, t b) = a + b
    fun t subtract(t a, t b) = a - b
    fun t divide(t a, t b) = a / b
    fun t multiply(t a, t b) = a * b
    
  } : NumType

  functor Doubler(number : NumType){
    type t = number.t
    fun t add(t a, t b) = number.add
    fun t double(t a) = add(a,a) 
  }

  struct IntDoubler = Doubler(Int)

  fun int main() = IntDoubler.double(2)
\end{lstlisting}
We first declare a signature NumType, which defines the number type. We declare it so
that there is a type t, that has four standard operations (add, subtract, divide
and multiply.).
\\
\\
We then declare a structure Int, which adheres to the NumType signature:
The add function in Int uses the built-in functionality of the plus operator in
Futhark, but could in principle be defined as whatever the programmer would
like \emph as long as the type checks correctly.

Then, the functor is defined: The functor takes a structure as an argument, and
can then refer to this structure in the declarations of the functor, using
dot-notation.
The functor contains a list of type-, function- and structure declarations, just
like the structure does.
However, none of these declarations has to be defined inside the functor:

Any of the declarations can be loaded from the functor's argument instead!

In the example above, the Doubler functor contains a function called
\texttt{double}. But \texttt{double} is not defined inside the functor, but will
instead become whatever the \texttt{double} function is defined as, inside the \texttt{number}
argument of the functor.
At the application of the Doubler functor, creating IntDoubler, we can
guarantee, that our types still hold. This is because the function definitions
in the functor has an abstract type\footnote{type t}, which is not instantiated
until the functor can get the type definition from the structure argument.

To boot, we can even bind a signature to functor if we want to. In that case, we
will have to create a functor, that returns a structure of signature
\textit{sig}, when it is applied.

\subsection{The reason for functors}
\label{subsec:reasonfunctors}

%% the reason is that WE CAN HAVE ALMOST POLYMORPHISM





\subsection{Implementing functors}
\label{subsec:implementing_functors}
Implementing a simple but 




